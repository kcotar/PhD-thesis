% intro
Observational studies and simulations show that stars in our Galaxy were not formed at the same time, but at multiple epochs in different places of the Galaxy \cite{2001ApJ...554.1044C, 2017ARA&A..55...59N}. One of the youngest building blocks of the Galaxy are open stellar clusters whose stars formed from the same molecular cloud of material \cite{2003ARA&A..41...57L} and therefore retain some properties of the original cloud they were made from. Stellar properties of an individual star can be separated into a kinematic and chemical component. The first is describing position and movement of a star in the Galaxy and the second its chemical composition and physical properties. During the evolution of the Galaxy, clusters that have lower number of members and are therefore gravitationally loosely bound, slowly evaporate because of different effect such as gravitational dynamical friction and ejection of stars during close intra-cluster stellar interactions. The latter is the result of a close gravitational interactions among members, where one of them can be ejected out of a cluster at high velocities \cite{2009MNRAS.396..570G, 2010MNRAS.402..105G, 2017MNRAS.470.3049R}. Such past members can on the sky be found even several degrees away from their main cluster body \cite{2007MNRAS.376L..29G, 2018MNRAS.473.4612K, 2019ApJ...884....6M}. The gravitational dynamical friction happens when a cluster moves trough regions of the Galaxy with higher destiny of stars \cite{2010MNRAS.401.2753B}. During such event, gravity of a cluster starts pulling seemingly stationary stars around it. Considering energy and momentum conservation law, we can conclude that a cluster will be slowed down for the same amount. This subsequently changes orbit of individual cluster members around the center of the Galaxy. Small orbital changes lead to a gradual expansion of a cluster volume that can be tracked until its individual stars blend with a general stellar field population, making them unrecognizable as past cluster members. The most prominent transitional features we can observe are compact cluster tidal tails \cite{2019AA...627A...4R, 2019AJ....157..115Y, 2019AA...621L...3M, 2019arXiv191206657Z} and loose extended halos of evaporated stars. The halos can be observed as a slowly decreasing over-density \cite{2002A&A...385..471C, 2004A&A...427..485B, 2019AA...627A.119C} of stars far from a denser main cluster core.

% cluster ages
Born from the same molecular cloud, open clusters are therefore ideal tracers of formation, assembly and evolutionary history of their host galaxies. Being influenced by external and internal processes, their lifetime is limited from about $100$~Myr to a few Gyr for the densest structures \cite{1998A&A...337..363P, 2013MNRAS.434.2509M}. Studies have shown that the dissolution time of a cluster depends mainly on its total mass, its radius and its galactic environment (density of the stars around it and on its galactic path). Many papers have so far dealt with the question of determining precise lifetime of stellar clusters before they completely dissipate into field population. The problem has been tackled using direct N-body simulations \cite{1998A&A...337..363P} where movement of individual stars was traced, and from direct observational data \cite{1971Ap&SS..13..300W, 1988IAUS..126..393W, 2019MNRAS.487.2385M, 2019A&A...623A.108B} using isochrone fits. Age distribution of open clusters within a distance of $1$~kpc from the Sun shows a cluster median lifetime of $200$~Myr. Their broad range in ages gives us a possibility of observing them at different evolutionary stages \cite{2006BASI...34..153C, 2007A&A...468..139P} before they blend \cite{2001A&A...366..827B} into a field stellar population. On the other hand, such a short lifetime (in comparison with a lifetime of a Sun-lie star) gives us a limited number of bounded clusters in the sky that can be studied at a given time.

% imf and mass segregation
The processes described above heavily depend on the mass of the cluster, its internal structure and mass distribution of its components. When star formation ceases, we are left with a wide range of stellar masses, their distribution can be described by an initial mass function (IMF) \cite{1955ApJ...121..161S, 1986FCPh...11....1S, 2003PASP..115..763C} that appears invariant among clusters and even stars in the field \cite{2001MNRAS.322..231K}. The initial spatial distribution of stars in young clusters may reflect the structure of parental molecular cloud \cite{2015MNRAS.448.1847H}. In many stellar clusters, the brightest and most massive stars are concentrated towards the center of a cluster, this state is usually attributed to mass segregation. Whether mass segregation occurs due to an evolutionary effect or it is of primordial origin is not yet entirely clear \cite{1998A&A...333..897R, 1998MNRAS.295..691B, 2002MNRAS.331..245D, 2003A&A...405..525B}. In the first case, massive stars formed elsewhere in the cluster and eventually sank to its center through the effect of two-body relaxations. In the second scenario, massive stars form preferentially in the central region of a cluster either by gas accretion due to their favorable location at the bottom of a gravitational potential well or through a coalescence process of less massive stars. The fact that mass segregation is also observed in young clusters might suggest that the second scenario is more likely, but the question is still under investigation \cite{2018MNRAS.473..849D}.

% binaries
%So far all discussion was focused on population of single stars. That is in contrast with observational data which suggests that a significant fraction of stars is found in binary, triple or even more complex stellar systems \cite{2010ApJS..190....1R, 2013ARA&A..51..269D, 2014AJ....147...87T}. 
Similarly to IMF, we can also define initial binary population (IBP) of observed sample of stars. Characterizing fraction of binaries in stellar clusters is of great importance for many fields of astrophysics. Since binaries are on average more massive than single stars they are thought to be good tracers of dynamical mass segregation \cite{2015arXiv151000099D}. Comparing observations with theoretical model for radial distribution of binary systems and their properties (mass and luminosity ratio, orbital period) distribution can be used to asses dynamical state of a cluster \cite{1999NewA....4..495K}.

\section{Open clusters in the \G\ era}
\label{sec:open_clust_gaia}
Since \G\ Data Processing and Analysis Consortium (DPAC) centers responsible for analysis of the \G\ observations started publishing publically available ready-to-use stellar properties, many new discovers and insights have been made. For the first time in history, we reliably know position, distance, complete spatial velocity (proper motion and radial velocity) and luminosity for millions of bright and dim stars all across the observable sky. One of fields in astronomy that heavily benefited from those new and improved measurements was research of open stellar clusters.

% looking for clusters, unreliable detection
A traditional historical way to look for open stellar clusters was through counting stars as they are apparently seen on the sky from Earths\' location and finding over-densities in those counts \cite{1988AJ.....95..108L, 2014A&A...568A..51S}. To perform a more robust selection, members are additionally filtered based on their apparent distance from the cluster center and motion vector \cite{2017A&A...601A..19G}. Using the latest second release of \G\ data \citep[DR2,][]{2018arXiv180409365G}, we can go beyond that and build upon the results achieved by the aforementioned methods. Complete \G\ information on stellar distance, kinematics and photometric measurements enables us to go beyond simple methodologies, to unravel even the faintest and sparsest components of open clusters. So far, many works have been published trying to refine parameters and membership information of long known open clusters \cite{2017A&A...601A..19G, 2018A&A...618A..93C, 2019A&A...627A..35C} and find new, less numerous or fainter clusters \cite{2019arXiv190904612B, 2019ApJS..245...32L, 2019JKAS...52..145S, 2019A&A...624A.126C, 2020arXiv200107122C}. Such thorough and improved investigation uncovered that many of the clusters listed in modern catalogues, initially discovered as apparent stellar overdensities, are no more than chance alignments of stars and not true physically bound clusters \cite{1998A&A...340..402B, 2000A&A...357..145C, 2016AJ....152....7H, 2018MNRAS.480.5242K, 2020A&A...633A..99C}.

% starosti zvezd za kopice in field
Precisely determined open cluster membership also enable better determine age of a cluster and its stars \cite{2018ApJ...863...65C}. When observing stars in any random stellar field it is notoriously hard to determine their ages as they remain unchanged for majority of its lifetime. While it is possible to obtain precise age estimates for some specific classes of stars \cite{2010ARA&A..48..581S}, an uniform methodology does not exist for all. This is much easier to determine for stars in a cluster as they are all of similar age. Stellar evolution and its cycle is determined mainly by the initial mass of a star. Observing stars of different masses in a cluster reveals its evolutionary stage and consequently also its age by comparing them to theoretical evolutionary models. The latest \G\ data enabled researches to compute stellar absolute magnitudes and colours that are commonly used in those comparisons \cite{2019MNRAS.487.2385M, 2019A&A...623A.108B, 2019A&A...631A.166K}.

% mass segregation
Another advantage of the newest \G\ dataset, with accurately determined distance to objects, is possibility to determine actual shape of a cluster, mass segregation of member stars and their gravitational potential from the first principles. Until recently that was possible only trough n-body simulations \cite{1987MNRAS.224..193T, 2016MNRAS.456.3757S, 2018MNRAS.473..849D} that were compared with available observational datasets. For distant stellar associations, the distance error bars are still larger than a cluster itself, but improvements are expected when \G\ EDR3 will be released in the second half of year 2020. Internal cluster dynamics can in those cases be inferred using precisely determined radial velocities whose uncertainty is not affected by stellar distance, but their apparent brightness. Reliable determination of stellar mass distribution in a cluster also depends on accurate classification of binary or higher multiplicity stars and their masses. By using pure \G\ information, they can be identified trough uncertain radial velocity \cite{2018RNAAS...2b..20E}, its temporal variation \cite{2019AJ....158..155B}, using precise absolute magnitudes \cite{2018ApJ...857..114W, 2018A&A...616A..10G, 2019MNRAS.487.2474} or exploring astrometric uncertainties \cite{2020arXiv200305467B} that indicate movement of a combined photocentre. 

\section{Chemical tagging}
Majority of previously described works in Section \ref{sec:open_clust_gaia} relies on a complete 6D positional and kinematics information to discover clusters and their sub-structures. Advances in observational techniques and data analysis enables us to go beyond kinematics information and include multidimensional chemical signature of stars -- procedure known as chemical tagging \cite{2002ARA&A..40..487F, 2010ApJ...721..582B}. So far a blind chemical tagging (without kinematics) that would delineate between cluster and field stars has not yet been demonstrated with great success, unless observed structure has obviously different chemical composition \cite{2016ApJ...833..262H}. A trait that it is not common to open stellar clusters formed at about the same time \cite{2019A&A...629A..34G}, but to galactic components formed at vastly different epochs \cite{2018A&A...619A.125A} of Galaxy formation.

Latest research showed that many, even unexpected, observational and data reduction issues still have to be thoroughly investigated and resolved, especially if data from different surveys are to be combined \cite{2019ARA&A..57..571J}. Studies suggest that open clusters might not be as homogeneous as thought before \cite{2016ApJ...817...49B, 2018MNRAS.473.4612K}, and abundances show traits of stellar evolution \cite{2015A&A...577A..47B, 2017ApJ...840...99D, 2018MNRAS.478..425B}. On top of that, main concerns of chemical tagging are abundance trends assumingly induced by spectral analysis \cite{2016ApJ...817...49B, 2019arXiv191208539C, 2020arXiv200103179B}. Observed trends depend on determined stellar physical parameters (i.e. \Teff\ and \vsin). Observed trends might be results of inadequate stellar models or actual stellar processes. To cope with this complexity and uncertainties, complex Bayesian models are being developed \cite{2016ApJ...817...49B, 2019ApJ...887...73C} in order to uncover and cluster abundance patterns.

\section{Peculiar stellar spectra}
Observational difficulties and data reduction issues are not the only obstacles that have to be controlled in order to perform successful blind chemical tagging. In every larger observational sample of stars, we can find a small sub-sample of stars with properties that deviate from the majority of randomly selected stellar population. Such stars can be identified by having kinematic properties vastly different than their local volume of space \cite{2010MNRAS.407.2241K, 2011ApJ...728..102W, 2017arXiv171003763C}, having distinctly unusual chemical abundances \cite{1974ARA&A..12..257P} or have spectral features that are seen in a small percentage of stars \cite{2010AJ....140.1758T, 2017ApJS..228...24T}. Among them, we also count sources on the sky whose observables show that they are a composite of two or more stellar components \cite{2010AJ....140..184M, 2018MNRAS.473.5043E, 2018MNRAS.476..528E}. In the opposite context, we commonly refer to majority of spectra as normal, but the delineation between classes is not strict and may depend on considered scientific question.

Every peculiarity in observed spectra could potentially influence derived stellar physical parameters and consequently also chemical abundances as they all depend on comparison between observed spectra and computed synthetic spectra that rely on physical models of stellar interior. As the exploration of galactic history, also known as the galactic archaeology, relies on normal stars for which their parameters can be reliable measured, we would like our set of analysed stars to be as clean as possible. The sets are usually delineated by the procedure known as classification. This refers to a procedure that sorts objects into categories and labels them according to their properties. Those labels can reflect actual physical properties and nature of stars or uncover unexpected features in observed data. Anyhow, both normal and peculiar stars are important for understanding large-scale galactic dynamics, internal structure of stars and their life cycle.

\section{Machine learning in large sky surveys}
In the past decades, we witnessed a major change in many areas of scientific research, including the filed of astronomy. New observational methods and advance scientific experiments are serving us ever increasing amounts of data that have to analysed by researchers. For example, astronomy have seen a shift from painstaking and time-consuming observations of a single star and its spectra to fiber-fed spectrographs that are able to simultaneously acquire spectra of hundreds to thousands stars in a comparable time. Among the studies dedicated to exploration of stellar objects we can find: Sloan Extension for Galactic
Understanding and Exploration (SEGUE), Apache Point Observatory Galactic Evolution Experiment (APOGEE), RAdial Velocity Experiment (RAVE), Gaia-ESO Survey (GES), Gaia spacecraft, GALactic Archaeology with Hermes (GALAH), 4-metre Multi-Object Spectrograph Telescope (4MOST), WHT Enhanced Area Velocity Explorer (WEAVE), and Funnel Web.

All mentioned surveys differ in the target selection methodology (magnitude and/or colour thresholds), properties of observed spectra (resolution, wavelength coverage etc.), and sky coverage, but all produce normal and peculiar spectra that have to identified and analysed. Because of a large number of observations, it is not feasible or justified to manually reduce and analyse all those spectra, therefore automatic pipelines have to be used. The first step in understanding observed spectroscopic data is a reduction pipeline, tailored to a specific telescopic and spectrograph setup \cite{2017MNRAS.464.1259K, 2019arXiv191202905A, 2020arXiv200204377S}, that will homogeneously prepare all spectra for further use and potentially add quality flags, warning a user that something might be wrong.

After the data have been prepared, machine learning approaches have been used in many different ways to extract further information from the observed spectra. The most important stellar properties in galactic archaeology are \Feh\ and individual chemical abundances. When having an access to a set of observations with correctly determined parameters, different approaches have been used to project them onto the whole dataset. Currently the most widely approaches are \TC\ \cite{2015ApJ...808...16N, buder2018} and Payne \citep{2019ApJ...879...69T} which fall into the category of generative approaches. Meaning that they, in order to infer parameters of analysed spectrum, internally generate a model spectrum that is compared to observations. Another large group of parameter determination machine learning approaches are regression algorithms that directly infer parameters defined by the training set. Of many existing regression approaches, currently the most explored is a use of various neural network architectures \citep{2015MNRAS.452..158Y, 2019MNRAS.483.3255L, 2020ApJ...891...23W, 2020arXiv200208390O}. The usage ranges from deep fully connected networks, convolutional networks, to autoencoder structures that first extract latent spectral features which are latter used in the regression procedure.

Moving to chemical tagging experiments, that build on top of previously determined stellar parameters and abundances, a shift from supervised to unsupervised machine learning algorithms is sensible as we almost newer do not know the desired result of a conducted experiment. In the literature, chemical tagging experiments often relay on existing unsupervised clustering methods such as K-means \cite{2015A&A...577A..47B, 2016ApJ...833..262H, 2018ApJ...860...70C}, DBSCAN \cite{2019MNRAS.487..871P}, t-SNE \cite{2018MNRAS.473.4612K, 2018A&A...619A.125A}, hierarchical clustering \cite{2017MNRAS.467.1140J, 2018A&A...618A..65B}, and other custom methodologies \cite{2019ApJ...887...73C}.

For some other surveys, such as \G, users do not have full access to raw observations, but have to trust to published stellar properties and their warning signs - such as uncertainties and quality flags - to filter out published data before performing any machine learning operations. 



% \section{Current and future challenges}
\section{Understanding observed data}
y research into open stellar clusters and spectroscopically peculiar stars consists of four related subtopics further explained in the text bellow: analysis of possible runaway stars ejected from clusters (based on \G\ and GALAH data), indication of multiple stellar systems in clusters (based on photometric surveys, RAVE and GALAH spectroscopic data) and investigation into possible Solar twins within observed clusters (GALAH spectroscopic data).
% members - drugačen pristop za iskanje daljno izvrženih članov - gledamo daleč okrog same kopice in integriramo orbite posameznih objektov glede na opazovano kopico, da se omeni kaj galah, na koncu dodamo še možnost njenega tagginga kot dodatna potrditev možnega članstva, mase iz isohrone na hr diagramu

\textbf{To study ejected stars} I will build upon available research that already defined membership possibilities and cluster parameters for about 3000 open clusters and stellar associations \cite{2013A&A...558A..53K} in the Galaxy. As they are mostly based on old datasets, it is beneficial to first test if their determined parameters, such as proper motion, still hold true or should be slightly changed. I will also use radial velocities to sift out outlying members. With the initial members of a cluster in place we can continue with the analysis on ejected stars located far away from their cluster center. Observed full 6D positional and kinematics vector for every star enables simulating their position in the Galaxy at different epochs. By integrating those properties we can model evolution of stellar clusters and their dissolution \cite{1998A&A...337..363P}. It has also been shown \cite{2000ApJ...544L.133H} that it is possible to retrace some of the nearest runaway stars back to its original cluster using old Hipparcos observations.

I will use the latest Gaia astrometric measurements supplemented with radial velocities measured by the RAVE, GALAH and GES spectroscopic survey as their observations were done for fainter sources, whose Gaia parameters will have large uncertainties for radial velocity or they won't exist at all. Plotting known members of the cluster on H-R diagram gives us its age by isochrone fitting \cite{2016ApJ...828...79J}. From the correct isochrone for observed cluster and photometric observations supplied by Gaia and other all sky survey we can infer mass of individual star. Mass of all observable members of a cluster will be used to compute its gravitational potential and possible influence on trajectories of nearby stars. I am interested in the stars outside cluster tidal radius which can be traced back to their cluster of origin. I plan to integrate the movement of the stars inside and outside cluster back in time to determine potential points where their orbits around the center of the Galaxy intersect. With this approach we can determine time of the ejection event. Events older than the cluster itself can be discarded as improbable cluster members and are more likely to belong to population of neighboring stars in the field. For the integration I plan to include simplified gravitational potential of the Galaxy \cite{2014MNRAS.437..351B} and of cluster members. % dodani zadnji stavki o integraciji in primerjavi starosti

Some of the stars that will be indicated as possible ejected members are also part of the GALAH survey. This gives us an additional possibility \cite{2018MNRAS.473.4612K} to check for similarities between their chemical composition and of the cluster, procedure also known as chemical tagging \cite{2002ARA&A..40..487F}.

% binaries - imajo pravi pm in razdaljo za kopico vendar ne štimajo druge stvari. Recimo rv je napačen in variabilen, napačna magnituda izseva in s tem položaj na HR diagramu, ki ima offset glede na predviden potek, modeliranje kakšen bi tak sistem moral biti da bi izseval toliko kot vidimo
\textbf{Binary and multiple stellar systems can hinder analysis} described in the previous section. From the observatories on and around Earth they are seen as point light sources and can not be visually separated. This leads to them having properly determined distance and proper motion that coincides with the cluster members but having mismatching parameters such as radial velocity and position on the H-R diagram \cite{2007A&A...473..829M}. Excess luminosity coming from two sources of comparable luminosity accounted as one positions them above the main sequence line, enabling us to separate them from normal stars. Such binaries can also be detected from spectroscopic observations where multiplications of the absorption features can be seen for both stars. They are also known as double-lined spectroscopic binaries (denoted as SB2).

Another type of binaries known as single-lined spectroscopic binaries (SB1) are not discernible from spectra, but solemnly from variations in their radial velocity. Accurate velocity measurements of cluster members with differences exceeding the cluster escape velocity can hint at the presence of SB1 with period up to several decades.

All large spectroscopic surveys employ data mining techniques to automatically detect resolvable SB2 candidates \cite{2010AJ....140..184M, 2017ApJS..228...24T} and unresolvable SB1 candidates \cite{2011AJ....141..200M} from their spectra. Precise look at the spectra of unflagged stars reveals that many of them still show signs of multiplied absorption features that were not picked out by automatic classification methods. Therefore there is still a room to improve the classification results and say something about the frequency of binaries and their properties in our the Galaxy. This is useful to infer the properties of initial mass function (IMF).

My plan is to search for signs of multiple stellar systems between possible open cluster members. As stated before I will first use Gaia proper motion and distance information to sift out cluster candidates. Those having photometric observations, radial velocities and spectroscopic data can be further analyzed. Members having mismatching radial velocities greater than the escape velocity of the cluster or being brighter than suggested by clusters' isochrone will be flagged as possible binaries and studied more thoroughly. Selection of the correct isochrone is made easier for the cluster as we know its age, metallicity and distance. Spectra  of the flagged objects will be analyzed and explained by simultaneously fitting \cite{2018MNRAS.473.5043E} synthetic spectra \cite{2005A&A...442.1127M} of multiple stars with different luminosity ratios to an observed spectrum. If needed, additional observations will be carried out at the Asiago observatory in Italy. % vmes vstavljen dodatek o isohronah in kovinskosti  

% solar siblings - morda najboljši most med vsemi je razdalja do zvezd. Z Soncu enakimi zvezdami lahko napovemo njihov točen izsev ki ga poznamo in iz tega tudi razdaljo do njih, to lahko uporabimo tudi za verifikacijo razdalj, še bolj zanimivo pa bi bilo videti možne variacije v kemični strukturi teh objektov, samo pomankanje določenih elemntov lahko nakazuje na prisotnost planetov kot na zemlji. Nekaj še o boljšem algoritmu za iskanje in velik dataset, razprostrt po velikem delu neba. Lahko gremo pa na peculiarity sončevega spektra, ker ima under abundance določenih elementov.

\textbf{Peculiarity can also be observed trough chemical composition of stars.} The best studied chemically peculiar star is our own Sun. When compared to spectroscopically similar stars, also named stellar twins (as defined by \cite{2017AN....338..442A}) it shows signs of underabundance of volatile chemical elements \cite{2009ApJ...704L..66M} that may hint to formation of solar system around observed star. When searching for solar twins we do not consider only chemical composition of the star but also their physical parameters. It would be interesting to see where in the Galaxy we can find solar twins observed by GALAH. If any of them is embedded inside a cluster of stars we can compare its chemical composition and analyze if and how it differs from other cluster members. Studies revealed that old open cluster M67 seems to contain several stars than could be regarded as solar twins \cite{2009MmSAI..80..125B, 2016MNRAS.463..696L}.

Solar twins are interesting for few other studies. From their observed luminosity and known luminosity of the Sun we can accurately determine their distance. Underabundace of volatile chemical elements in solar twins can be correlated with the presence of planetary systems around selected stars. Fraction of studied stars in the GALAH survey have or will be studied by ongoing K2 \cite{2014PASP..126..398H} and future TESS \cite{2014SPIE.9143E..20R} spaceborn missions that are searching for planets around other stars. Matching known stars with planetary systems with our GALAH sample might reveal abundance patterns that can hint at the presence of rocky or gaseous planets.
% še kaj o Keppler, K2 in TESS na tem mestu in tu 

To search for solar twins, I will use raw spectral data acquired in the scope of the GALAH survey. Reference Solar spectrum will be created from acquired twilight flats and final stacked spectra compared to other Solar spectra published in literature to analyze possible deviations from it. The first selection of possible Solar twins is refined on the basis of Cannon \cite{2015ApJ...808...16N} determined stellar parameters. For this selection I will first identify possible offsets between Solar parameters and the ones determined for observed flats and Gaia FGK standard stars \cite{2014A&A...564A.133J, 2015A&A...582A..49H} with Solar like parameters. Direct spectral comparison of selected GALAH objects and our reference Solar spectrum will be computed only for the regions of selected absorption lines. Similarity determined by multiple distance metrices was found to be highly correlated with a signal to noise ratio of the observed spectrum. To improve similarity computation I plan to use a Gaussian process \cite{2006gpml.book.....R} model of observed noise described with two radial kernel functions: one describing noise and the other differences in continuum levels. Model parameters can be fitted using Bayesian MCMC approach \cite{2013PASP..125..306F}. Final selection of Solar twins will be based on their similarity towards solar spectrum.