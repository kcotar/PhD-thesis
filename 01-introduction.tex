% intro
Observational studies and simulations show that stars in our Galaxy were not formed at the same time, but at multiple epochs in different places of the Galaxy \citep{2001ApJ...554.1044C, 2017ARA&A..55...59N}. One of the building blocks of the Galaxy are open stellar clusters whose stars formed from the same molecular cloud of material \citep{2003ARA&A..41...57L} and thus retain some properties of the original cloud they were made from. Properties can be separated into kinematic and chemical component. The first is describing position and movement of a star and the second its chemical composition. During the evolution of the Galaxy, clusters that are gravitationally loosely bound, slowly evaporate because of different effect such as close stellar interactions and gravitational dynamical friction. This happens when a cluster moves trough regions of the Galaxy with higher destiny of stars \citep{2010MNRAS.401.2753B}. During such event, gravity of a cluster starts pulling seemingly stationary stars around it. Considering energy and momentum conservation law, we can conclude that a cluster will be slowed down for the same amount. This also changes its orbit around the center of the Galaxy. During the interaction cluster members can also  get ejected out of a cluster and become part of the general stellar field population. This effect is also known as stellar evaporation.

Stellar clusters can thus be used as powerful tracers of formation, assembly and evolutionary history of their host galaxies. One of the critical remaining issues essential to understanding the early evolution of star clusters is the question of precisely determining how and when stars form within them \citep{2010RSPTA.368..693D}.

% novo - starosti zvezd za kopice in field
When observing stars in any random stellar field it is notoriously hard to determine their ages. While it is possible to obtain precise age estimates for some specific classes of stars \citep{2010ARA&A..48..581S}, a uniform methodology does not exist for all. This is much easier to determine for stars in a cluster as they are all of similar age. Stellar evolution and its cycle is determined mainly by the mass of a star. Observing stars of different masses in a cluster reveals its evolutionary stage and consequently also its age.



\section{Open clusters in the \G\ era}
Since teams responsible for analysis of the \G\ observations started producing and publishing publically available kinematic and photometric information for stars across the whole sky, many new discovers and insights have been made. For the first time in history, we reliably know distance, movement and luminosity for millions of bright and dim stars. One of fields in astronomy that heavily benefited from those new and improved measurements was research of open stellar clusters.

A traditional historical way to look for open stellar clusters was through counting stars as they are apparently seen on the sky from Earths\' location and finding over-densities in those counts \citep{1988AJ.....95..108L, 2014A&A...568A..51S}. To perform a more robust selection, members are additionally filtered based on their apparent distance from the cluster center and motion vector \citep{2017A&A...601A..19G}. With the new and improved all sky datasets we can go beyond that and build upon the results achieved with the mentioned methods. 


The main advantage of the newest Gaia dataset, with accurately determined distance to the objects, is possibility to determine actual shape of the cluster, mass segregation and its gravitational potential from the first principles. Until recently that was possible only trough n-body simulations \citep{1987MNRAS.224..193T, 2016MNRAS.456.3757S, 2018MNRAS.473..849D}. For distant stellar associations, the distance error bar will still be larger than cluster itself, but we can point to dynamics within the cluster with precisely determined radial velocities. 

To determine full stellar velocity vector and position two different types of observations are needed. Stellar proper motion is determined from consecutive images of the same stellar field on which nearby stars show some movement in comparison to apparently stationary stars far away from us. The radial component of the velocity can only be extracted from spectroscopic observations by determining the Doppler shift of the absorption lines in the stellar spectrum. Until recently our knowledge of those parameters for the majority of stars was only approximate and associated uncertainties were very large. This will change on 25 April when Esa will publish the second data release of Gaia \cite{2001A&A...369..339P, 2008IAUS..248..217L, 2015A&A...576A..74D} parameters (Gaia DR2) determined from its observations between 25 July 2014 and 23 May 2016. A newly released dataset will consist of more than 1.3 billion stars with accurately determined distance and proper motion. For a faint star (G$\sim$17) at the distance of 1 kpc, it is expected that its distance error will be less than 10 \% and the accuracy of the proper motion in the order of 1 km/s. About 6 million of those stars brighter than magnitude $13$ in the G band will also contain information about their median radial velocity. The accuracy of the radial velocity for the brightest sources will be in the order of $200$ - $300$ m/s and about $2$ km/s for the faintest sources. This is comparable to an open cluster escape velocity and thus not suitable for studies of motion within cluster. These 6D phase space data, independent of modeling and isochrone fitting, in combination with detailed stellar chemical abundances, will help us to disentangle the history of Milky Way formation.

To complement and improve radial velocity measurements performed by Gaia, multiple different large all sky high resolution spectroscopic surveys can be used. Accuracies of their measurements can get well bellow $100$ m/s even for sources much fainter than the limiting magnitude of the Gaia satellite. The largest available sets of spectroscopic data were acquired by fiber-fed spectrographs that are becoming popular in the recent years as they can analyze spectra from few a hundred up to a thousand stars at the same time. Among those dedicated to stellar objects are: Sloan Extension for Galactic Understanding and Exploration (SEGUE \cite{2009AJ....137.4377Y}), Apache Point Observatory Galactic Evolution Experiment (APOGEE \cite{2017AJ....154...94M}), RAdial Velocity Experiment (RAVE \cite{2017AJ....153...75K}), Gaia-ESO Survey (GES \cite{2012Msngr.147...25G}) and GALactic Archaeology with HERMES (GALAH \cite{2017MNRAS.465.3203M}).

The goal of GALAH is to disentangle the formation history of the Milky Way, using fossil remnants of ancient star formation and accretion events which have been disrupted and are now dispersed around the Galaxy. Recent studies of stellar chemical abundances in individual open clusters show that their abundance distributions are homogeneous to the level at which they can be measured \cite{2016ApJ...817...49B, 2016ApJ...833..262H}, and their abundances are different from cluster to cluster (conversely, globular clusters display inhomogeneities \cite{2007MNRAS.377..335B, 2012A&ARv..20...50G}). GALAH will measure up to 30 elemental abundances from 7 independent element groups, thereby obtaining enough independent cells in the multi-dimensional chemical abundance feature space (also known as C-space), in which stars from chemically homogeneous stellar associations will lie close together \cite{2012ASPC..458..393F}. Acquired level of accuracy and the amount of elemental abundance information by far surpasses any existing spectroscopic survey.


\section{Chemical tagging}
% dissipation time of the cluster
Many papers so far have dealt with the question of determining lifetime of stellar clusters before they dissipate into field population. The problem has been tackled using direct N-body simulations \citep{1998A&A...337..363P} where movement of individual stars was traced and from observational data \citep{1971Ap&SS..13..300W, 1988IAUS..126..393W}. Age distribution of open clusters within a distance of $1$ kiloparsecs from the Sun shows a cluster median lifetime of $200$ million years. Such a short time of existence (in comparison with a lifetime of a star) gives us a limited number of bounded clusters in the sky that can be studied at a given time. Studies have shown that the dissolution time of a cluster depends mainly on its total mass, its radius and its galactic environment (density of the stars around it and on its galactic path). 

% IMF and mass segregation
The processes described above heavily depend on the mass of the cluster, its internal structure and mass distribution of its components. When star formation ceases we are left with a wide range of stellar masses, their distribution can be described by an initial mass function (IMF) \cite{1955ApJ...121..161S} that appears invariant among clusters and even stars in the field. The initial spatial distribution of stars in young clusters may reflect the structure of parental molecular cloud \cite{2015MNRAS.448.1847H}. As cluster evolve, its internal structure is shaped by gravitational interactions between cluster members and by external tidal effects. 

In many stellar clusters, the brightest and most massive stars are concentrated towards the center of a cluster, this state is usually attributed to mass segregation. Whether mass segregation occurs due to an evolutionary effect or it is of primordial origin is not yet entirely clear. In the first case, massive stars formed elsewhere in the cluster and eventually sank to its center through the effect of two-body relaxations. In the second scenario, massive stars form preferentially in the central region of a cluster either by gas accretion due to their favorable location at the bottom of a gravitational potential well or through a coalescence process of less massive stars. The fact that mass segregation is also observed in young clusters might suggest that the second scenario is more likely, but the question is still under investigation \cite{2018MNRAS.473..849D}.

% binaries
So far all discussion was focused on population of single stars. That is in contrast with observational data which suggests that a significant fraction of stars is found in binary, triple or even more complex stellar systems. Same as IMF we can also define initial binary population (IBP) of observed sample of stars. Characterizing fraction of binaries in stellar clusters is of great importance for many fields of astrophysics. Since binaries are on average more massive than single stars they are thought to be good tracers of dynamical mass segregation \cite{2015arXiv151000099D}. Comparing observations with theoretical model for radial distribution of binary systems and their properties (mass and luminosity ratio, orbital period) distribution can be used to asses dynamical state of a cluster \cite{1999NewA....4..495K}.

Binary stars are quite often classified as peculiar objects \cite{2017ApJS..228...24T} among normal stars. In every larger observational sample of stars we can find a small sub sample of stars with unusual properties. They can be expressed as positional and kinematic outliers in their local volume of space \cite{2010MNRAS.407.2241K, 2011ApJ...728..102W, 2017arXiv171003763C}, having distinctly unusual chemical abundances \cite{1974ARA&A..12..257P} or having having spectral features that are seen in a small percentage of stars \cite{2010AJ....140.1758T}.

\section{Current and future challenges}

My research into open stellar clusters consists of three related subtopics further explained in the text bellow: analysis of possible runaway stars ejected from clusters (based on Gaia, RAVE and GALAH data), indication of multiple stellar systems in clusters (based on photometric surveys, RAVE and GALAH spectroscopic data) and investigation into possible Solar twins within observed clusters (GALAH spectroscopic data).
% members - drugačen pristop za iskanje daljno izvrženih članov - gledamo daleč okrog same kopice in integriramo orbite posameznih objektov glede na opazovano kopico, da se omeni kaj galah, na koncu dodamo še možnost njenega tagginga kot dodatna potrditev možnega članstva, mase iz isohrone na hr diagramu

\textbf{To study ejected stars} I will build upon available research that already defined membership possibilities and cluster parameters for about 3000 open clusters and stellar associations \cite{2013A&A...558A..53K} in the Galaxy. As they are mostly based on old datasets, it is beneficial to first test if their determined parameters, such as proper motion, still hold true or should be slightly changed. I will also use radial velocities to sift out outlying members. With the initial members of a cluster in place we can continue with the analysis on ejected stars located far away from their cluster center. Observed full 6D positional and kinematics vector for every star enables simulating their position in the Galaxy at different epochs. By integrating those properties we can model evolution of stellar clusters and their dissolution \cite{1998A&A...337..363P}. It has also been shown \cite{2000ApJ...544L.133H} that it is possible to retrace some of the nearest runaway stars back to its original cluster using old Hipparcos observations.

I will use the latest Gaia astrometric measurements supplemented with radial velocities measured by the RAVE, GALAH and GES spectroscopic survey as their observations were done for fainter sources, whose Gaia parameters will have large uncertainties for radial velocity or they won't exist at all. Plotting known members of the cluster on H-R diagram gives us its age by isochrone fitting \cite{2016ApJ...828...79J}. From the correct isochrone for observed cluster and photometric observations supplied by Gaia and other all sky survey we can infer mass of individual star. Mass of all observable members of a cluster will be used to compute its gravitational potential and possible influence on trajectories of nearby stars. I am interested in the stars outside cluster tidal radius which can be traced back to their cluster of origin. I plan to integrate the movement of the stars inside and outside cluster back in time to determine potential points where their orbits around the center of the Galaxy intersect. With this approach we can determine time of the ejection event. Events older than the cluster itself can be discarded as improbable cluster members and are more likely to belong to population of neighboring stars in the field. For the integration I plan to include simplified gravitational potential of the Galaxy \cite{2014MNRAS.437..351B} and of cluster members. % dodani zadnji stavki o integraciji in primerjavi starosti

Some of the stars that will be indicated as possible ejected members are also part of the GALAH survey. This gives us an additional possibility \cite{2018MNRAS.473.4612K} to check for similarities between their chemical composition and of the cluster, procedure also known as chemical tagging \cite{2002ARA&A..40..487F}.

% binaries - imajo pravi pm in razdaljo za kopico vendar ne štimajo druge stvari. Recimo rv je napačen in variabilen, napačna magnituda izseva in s tem položaj na HR diagramu, ki ima offset glede na predviden potek, modeliranje kakšen bi tak sistem moral biti da bi izseval toliko kot vidimo
\textbf{Binary and multiple stellar systems can hinder analysis} described in the previous section. From the observatories on and around Earth they are seen as point light sources and can not be visually separated. This leads to them having properly determined distance and proper motion that coincides with the cluster members but having mismatching parameters such as radial velocity and position on the H-R diagram \cite{2007A&A...473..829M}. Excess luminosity coming from two sources of comparable luminosity accounted as one positions them above the main sequence line, enabling us to separate them from normal stars. Such binaries can also be detected from spectroscopic observations where multiplications of the absorption features can be seen for both stars. They are also known as double-lined spectroscopic binaries (denoted as SB2).

Another type of binaries known as single-lined spectroscopic binaries (SB1) are not discernible from spectra, but solemnly from variations in their radial velocity. Accurate velocity measurements of cluster members with differences exceeding the cluster escape velocity can hint at the presence of SB1 with period up to several decades.

All large spectroscopic surveys employ data mining techniques to automatically detect resolvable SB2 candidates \cite{2010AJ....140..184M, 2017ApJS..228...24T} and unresolvable SB1 candidates \cite{2011AJ....141..200M} from their spectra. Precise look at the spectra of unflagged stars reveals that many of them still show signs of multiplied absorption features that were not picked out by automatic classification methods. Therefore there is still a room to improve the classification results and say something about the frequency of binaries and their properties in our the Galaxy. This is useful to infer the properties of initial mass function (IMF).

My plan is to search for signs of multiple stellar systems between possible open cluster members. As stated before I will first use Gaia proper motion and distance information to sift out cluster candidates. Those having photometric observations, radial velocities and spectroscopic data can be further analyzed. Members having mismatching radial velocities greater than the escape velocity of the cluster or being brighter than suggested by clusters' isochrone will be flagged as possible binaries and studied more thoroughly. Selection of the correct isochrone is made easier for the cluster as we know its age, metallicity and distance. Spectra  of the flagged objects will be analyzed and explained by simultaneously fitting \cite{2018MNRAS.473.5043E} synthetic spectra \cite{2005A&A...442.1127M} of multiple stars with different luminosity ratios to an observed spectrum. If needed, additional observations will be carried out at the Asiago observatory in Italy. % vmes vstavljen dodatek o isohronah in kovinskosti  

% solar siblings - morda najboljši most med vsemi je razdalja do zvezd. Z Soncu enakimi zvezdami lahko napovemo njihov točen izsev ki ga poznamo in iz tega tudi razdaljo do njih, to lahko uporabimo tudi za verifikacijo razdalj, še bolj zanimivo pa bi bilo videti možne variacije v kemični strukturi teh objektov, samo pomankanje določenih elemntov lahko nakazuje na prisotnost planetov kot na zemlji. Nekaj še o boljšem algoritmu za iskanje in velik dataset, razprostrt po velikem delu neba. Lahko gremo pa na peculiarity sončevega spektra, ker ima under abundance določenih elementov.

\textbf{Peculiarity can also be observed trough chemical composition of stars.} The best studied chemically peculiar star is our own Sun. When compared to spectroscopically similar stars, also named stellar twins (as defined by \cite{2017AN....338..442A}) it shows signs of underabundance of volatile chemical elements \cite{2009ApJ...704L..66M} that may hint to formation of solar system around observed star. When searching for solar twins we do not consider only chemical composition of the star but also their physical parameters. It would be interesting to see where in the Galaxy we can find solar twins observed by GALAH. If any of them is embedded inside a cluster of stars we can compare its chemical composition and analyze if and how it differs from other cluster members. Studies revealed that old open cluster M67 seems to contain several stars than could be regarded as solar twins \cite{2009MmSAI..80..125B, 2016MNRAS.463..696L}.

Solar twins are interesting for few other studies. From their observed luminosity and known luminosity of the Sun we can accurately determine their distance. Underabundace of volatile chemical elements in solar twins can be correlated with the presence of planetary systems around selected stars. Fraction of studied stars in the GALAH survey have or will be studied by ongoing K2 \cite{2014PASP..126..398H} and future TESS \cite{2014SPIE.9143E..20R} spaceborn missions that are searching for planets around other stars. Matching known stars with planetary systems with our GALAH sample might reveal abundance patterns that can hint at the presence of rocky or gaseous planets.
% še kaj o Keppler, K2 in TESS na tem mestu in tu 

To search for solar twins, I will use raw spectral data acquired in the scope of the GALAH survey. Reference Solar spectrum will be created from acquired twilight flats and final stacked spectra compared to other Solar spectra published in literature to analyze possible deviations from it. The first selection of possible Solar twins is refined on the basis of Cannon \cite{2015ApJ...808...16N} determined stellar parameters. For this selection I will first identify possible offsets between Solar parameters and the ones determined for observed flats and Gaia FGK standard stars \cite{2014A&A...564A.133J, 2015A&A...582A..49H} with Solar like parameters. Direct spectral comparison of selected GALAH objects and our reference Solar spectrum will be computed only for the regions of selected absorption lines. Similarity determined by multiple distance metrices was found to be highly correlated with a signal to noise ratio of the observed spectrum. To improve similarity computation I plan to use a Gaussian process \cite{2006gpml.book.....R} model of observed noise described with two radial kernel functions: one describing noise and the other differences in continuum levels. Model parameters can be fitted using Bayesian MCMC approach \cite{2013PASP..125..306F}. Final selection of Solar twins will be based on their similarity towards solar spectrum.