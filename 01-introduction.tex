 % intro
Observational studies and simulations show that stars in our Galaxy were not formed at the same time, but at multiple epochs in different places of the Galaxy \cite{2001ApJ...554.1044C, 2017ARA&A..55...59N}. One of the youngest building blocks of the Galaxy are open stellar clusters whose stars formed from the same molecular cloud of material \cite{2003ARA&A..41...57L} and therefore retain some properties of the original cloud they were made from. Stellar properties of an individual star can be separated into a kinematic and chemical component. The first is describing the position and movement of a star in the Galaxy. The second gives information about its chemical composition and physical properties. During the evolution of the Galaxy, clusters that have a lower number of members and are therefore gravitationally loosely bound, slowly evaporate because of different effects such as gravitational dynamical friction and ejection of stars during close intra-cluster stellar interactions. The latter is a result of close gravitational interactions among members, where one of them can be ejected out of a cluster at high velocity \cite{2009MNRAS.396..570G, 2010MNRAS.402..105G, 2017MNRAS.470.3049R}. Such past members can on the sky be found even several degrees away from their main cluster body \cite{2007MNRAS.376L..29G, 2018MNRAS.473.4612K, 2019ApJ...884....6M}. The gravitational dynamical friction happens when a cluster moves trough regions of the Galaxy with a higher density of stars \cite{2010MNRAS.401.2753B}. During such event, the gravity of a cluster starts pulling seemingly fixed stars around it. Considering energy and momentum conservation law, we can conclude that a cluster will be slowed down for the same amount. This slowing subsequently changes the orbit of individual cluster members around the centre of the Galaxy. Small orbital changes lead to a gradual expansion of a cluster volume that can be tracked until its individual stars blend with a general stellar field population, making them unrecognizable as past cluster members. The most prominent transitional features we can observe are compact cluster tidal tails \cite{2019AA...627A...4R, 2019AJ....157..115Y, 2019AA...621L...3M, 2019arXiv191206657Z} and lose extended halos of evaporated stars. The halos can be observed as a slowly decreasing over-density \cite{2002A&A...385..471C, 2004A&A...427..485B, 2019AA...627A.119C} of stars far from a denser central cluster core.

% cluster ages
Born from the same molecular cloud, open clusters are therefore ideal tracers of formation, assembly and evolutionary history of their host galaxies. Being influenced by external and internal processes, their lifetime is limited from about $100$~Myr to a few Gyr for the densest structures \cite{1998A&A...337..363P, 2013MNRAS.434.2509M}. Studies have shown that the dissolution time of a cluster depends mainly on its total mass, its radius and its galactic environment (density of the stars around it and on its galactic path). Many papers have so far dealt with the question of determining the precise lifetime of stellar clusters before they completely dissipate into field population. The problem has been tackled using direct N-body simulations \cite{1998A&A...337..363P} where the movement of individual stars was traced, and from direct observational data \cite{1971Ap&SS..13..300W, 1988IAUS..126..393W, 2019MNRAS.487.2385M, 2019A&A...623A.108B} using isochrone fits. Age distribution of open clusters within a distance of $1$~kpc from the Sun shows a cluster median lifetime of $200$~Myr. Their broad range in ages gives us a possibility of observing them at different evolutionary stages \cite{2006BASI...34..153C, 2007A&A...468..139P} before they blend \cite{2001A&A...366..827B} into a field stellar population. On the other hand, such a short lifetime (in comparison with a lifetime of a Sun-like star) gives us a limited number of bounded clusters in the sky that can be studied at a given time.

% imf and mass segregation
The processes described above heavily depend on the mass of a cluster, its internal structure and mass distribution of its components. When star formation ceases, we are left with a wide range of stellar masses. Their distribution can be described by an initial mass function (IMF) \cite{1955ApJ...121..161S, 1986FCPh...11....1S, 2003PASP..115..763C} that appears invariant among clusters and even stars in the field \cite{2001MNRAS.322..231K}. The initial spatial distribution of stars in young clusters may reflect the structure of parental molecular cloud \cite{2015MNRAS.448.1847H}. In many stellar clusters, the brightest and most massive stars are concentrated towards the centre of a cluster, this state is usually attributed to mass segregation. Whether mass segregation occurs due to an evolutionary effect or it is of primordial origin is not yet entirely clear \cite{1998A&A...333..897R, 1998MNRAS.295..691B, 2002MNRAS.331..245D, 2003A&A...405..525B}. In the first case, massive stars are formed all around the cluster volume and eventually sink to its centre through the effect of two-body relaxations. In the second scenario, massive stars form preferentially in the central region of a cluster either by gas accretion due to their favourable location at the bottom of a gravitational potential well or through a coalescence process of less massive stars. The fact that mass segregation is also observed in young clusters might suggest that the second scenario is more likely that the evolutionary mass segregation, but the question is still under investigation \cite{2018MNRAS.473..849D}.

% binaries 
Similarly to the IMF, we can also define the initial binary population (IBP) of the observed sample of stars. Characterizing fraction of binaries in stellar clusters is of great importance for many fields of astrophysics. Since binaries are on average more massive than single stars, they are thought to be useful tracers of dynamical mass segregation \cite{2015arXiv151000099D}. Comparing observations with a theoretical model for the radial distribution of binary systems and their properties (mass and luminosity ratio, orbital period) distribution can be used to assess the dynamical state of a cluster \cite{1999NewA....4..495K}.

\section{Open clusters in the \G\ era}
\label{sec:open_clust_gaia}
Since \G\ Data Processing and Analysis Consortium (DPAC) centres responsible for analysis of the \G\ observations started publishing publicly available ready-to-use stellar properties, many new discovers and insights have been made. For the first time in history, we reliably know the position, distance, complete spatial velocity (proper motion and radial velocity) and luminosity for millions of bright and dim stars all across the observable sky. Their accuracy heavily depends on the source brightness. For example, published \G\ parallax uncertainties are in the range from $0.04$ to $0.7$~mas for the faintest sources. Similarly, radial velocities are precisely known in the range from 300~\ms to 3~\kms (further details are given in Section \ref{sec:gaia_data}).

One of the fields in astronomy that massively benefited from those new and improved measurements was a research of open stellar clusters.

% looking for clusters, unreliable detection
A traditional historical way to look for open stellar clusters was through counting stars as they are seen on the sky from Earths' location and finding over-densities in those counts \cite{1988AJ.....95..108L, 2014A&A...568A..51S}. To perform a more robust selection, members were additionally filtered based on their apparent distance from the cluster centre and their motion vectors \cite{2017A&A...601A..19G}. Using the latest second release of \G\ data \citep[DR2,][]{2018A&A...616A...1G}, we can go beyond that and build upon the results achieved by the methods mentioned above. Complete \G\ information on stellar distance, kinematics and photometric measurements enable us to go beyond simple methodologies, to unravel even the faintest and sparsest components of open clusters. So far, many works have been published trying to refine parameters, and membership information of long known open clusters \cite{2017A&A...601A..19G, 2018A&A...618A..93C, 2019A&A...627A..35C} and find new, less numerous or fainter clusters \cite{2019arXiv190904612B, 2019ApJS..245...32L, 2019JKAS...52..145S, 2019A&A...624A.126C, 2020arXiv200107122C}. Such thorough and the improved investigation uncovered that many of the clusters listed in modern catalogues, initially discovered as apparent stellar overdensities, are no more than chance alignments of stars and not true physically bound clusters \cite{1998A&A...340..402B, 2000A&A...357..145C, 2016AJ....152....7H, 2018MNRAS.480.5242K, 2020A&A...633A..99C}.

% starosti zvezd za kopice in field
Precisely determined open cluster membership also enables accurate age determination of a cluster and its stars \cite{2018ApJ...863...65C}. When observing stars in any random stellar field, it is notoriously hard to determine their ages as they remain unchanged for the majority of their lifetime. While it is possible to obtain precise age estimates for some specific classes of stars \cite{2010ARA&A..48..581S}, a uniform methodology does not exist for all. Age determination is much easier for stars in a cluster as they are all of a very similar age. Stellar evolution and its cycle are determined mainly by the initial mass of a star. Observing stars of different masses in a cluster reveals its evolutionary stage and consequently also its age by comparing them to theoretical evolutionary models. The latest \G\ data enabled researchers to compute stellar absolute magnitudes and colours that are commonly used in those comparisons \cite{2019MNRAS.487.2385M, 2019A&A...623A.108B, 2019A&A...631A.166K}.

% mass segregation
Another advantage of the newest \G\ dataset, with accurately determined distance to objects, is the possibility of determining the actual shape of a cluster, mass segregation of member stars and their gravitational potential from the first principles. Until recently that was possible only through n-body simulations \cite{1987MNRAS.224..193T, 2016MNRAS.456.3757S, 2018MNRAS.473..849D} that were compared with available observational datasets. For distant stellar associations, the distance error bars are still larger than a cluster itself, but improvements are expected when \G\ EDR3 will be released in the second half of the year 2020 or later. Internal cluster dynamics can in those cases be inferred using precisely determined radial velocities and proper motion vectors whose uncertainty is not affected by stellar distance, but their apparent brightness. Deviations (overtaking or lagging) from the mean cluster velocity can be therefore be used to assess in which direction, away from the mass centre, stars are moving. This velocity deviation consequently also indicates its position in a cluster. Reliable determination of stellar mass distribution in a cluster also depends on accurate classification of binary or higher multiplicity stars and their masses. By using pure \G\ information, they can be identified as stars with higher radial velocity uncertainty \cite{2018RNAAS...2b..20E}, by its temporal variation in future data releases \cite{2019AJ....158..155B}, using precise absolute magnitudes \cite{2018ApJ...857..114W, 2018A&A...616A..10G, 2019MNRAS.487.2474C} or by exploring astrometric uncertainties \cite{2020arXiv200305467B} that indicate movement of their combined photocentre. 

\section{Chemical tagging}
\label{sec:open_clustesr_tagging}
Majority of previously described works in Section \ref{sec:open_clust_gaia} relies on a complete 6D positional and kinematics information to discover clusters and their sub-structures. Advances in observational techniques and data analysis enables us to go beyond kinematics information and include a multidimensional chemical signature of stars -- the procedure known as chemical tagging \cite{2002ARA&A..40..487F, 2010ApJ...721..582B}. So far a blind chemical tagging (without kinematics) that would delineate between cluster and field stars has not yet been demonstrated with great success unless the observed structure has obviously different chemical composition \cite{2016ApJ...833..262H}. A trait that it is not common to open stellar clusters formed at about the same time \cite{2019A&A...629A..34G}, but to galactic components formed at vastly different epochs \cite{2018A&A...619A.125A} of Galaxy formation.

The measured chemical signature of a star gives us its composition in the outer-most layer, where it remains unchanged for young stars. Its composition is, therefore, a direct reflection of a medium from which it was built. Galaxy started evolving from dust made predominantly from hydrogen and helium which slowly got polluted by evolved stars during their final evolutionary stages. This gradual enrichment is nowadays observable as gradients of chemical abundances in radial and vertical directions of Galaxy. Internal gravitational mixing causes blending of those signatures as stars move to different orbits. Already complicated chemical structure of Galaxy is in some regions interrupted by newly created dense stellar structures -- open clusters -- which we would like to discover by the above-described procedure. Because of multiple gradual processes going on in Galaxy, the chemical tagging so far does not have clearly defined cuts to separate galactic components.

The latest research showed that many, even unexpected, observational and data reduction issues still have to be thoroughly investigated and resolved, especially if data from different surveys are to be combined \cite{2019ARA&A..57..571J}. Studies suggest that open clusters might not be as homogeneous as thought before \cite{2016ApJ...817...49B, 2018MNRAS.473.4612K}, and abundances show traits of stellar evolution \cite{2015A&A...577A..47B, 2017ApJ...840...99D, 2018MNRAS.478..425B}. On top of that, the main concerns of chemical tagging are abundance trends assumingly induced by spectral analysis \cite{2016ApJ...817...49B, 2019arXiv191208539C, 2020arXiv200103179B}. Observed trends depend on determined stellar physical parameters (i.e. \Teff\ and \vsin) and might be results of inadequate stellar models or actual stellar processes. To cope with this complexity and uncertainties, complex Bayesian models are being developed \cite{2016ApJ...817...49B, 2019ApJ...887...73C} in order to uncover and cluster abundance patterns. With this in mind, many work and validation still have to be done until large surveys are fully ready for blind chemical tagging experiments.

\section{Peculiar stellar spectra}
Observational difficulties and data reduction issues are not the only obstacles that have to be controlled in order to perform successful blind chemical tagging. In every larger observational sample of stars, we can find a small sub-sample of stars with properties that deviate from the majority of the randomly selected stellar population. Such stars can be identified by having kinematic properties vastly different than their local volume of space \cite{2010MNRAS.407.2241K, 2011ApJ...728..102W, 2017arXiv171003763C}, having distinctly unusual chemical abundances \cite{1974ARA&A..12..257P} or have spectral features that are seen in a small percentage of stars \cite{2010AJ....140.1758T, 2017ApJS..228...24T}. Among them, we also count sources on the sky whose observables show that they are a composite of two or more stellar components \cite{2010AJ....140..184M, 2018MNRAS.473.5043E, 2018MNRAS.476..528E}. During the early stages of stellar evolution, stars could be found surrounded by the optically thin material from which they formed \cite{2007prpl.conf..361A}. Such peculiar stars, commonly found in young open clusters \cite{2012AJ....143...61N, 2015A&A...581A..52T}, exhibit additional emission features in their spectra. In the opposite context, we commonly refer to the majority of spectra as normal, but the delineation between classes is not strict and may depend on a considered scientific question.

Every peculiarity in observed spectra could potentially influence derived stellar physical parameters and consequently also chemical abundances as they all depend on the comparison between observed spectra and computed synthetic spectra that rely on physical models of stellar interior. As the exploration of galactic history, also known as the galactic archaeology, relies on normal stars for which their parameters can be reliably measured, we would like our set of analysed stars to be as clean as possible. The sets are usually delineated by the procedure known as classification. This refers to a procedure that sorts objects into categories and labels them according to their properties. Those labels can reflect actual physical properties and nature of stars or uncover unexpected features in observed data. Anyhow, both normal and peculiar stars are essential for understanding large-scale galactic dynamics, the internal structure of stars and their life cycle.

\section{Machine learning in large sky surveys}
In the past decades, we witnessed a significant change in many areas of scientific research, including the filed of astronomy. New observational methods and advance scientific experiments are serving us ever-increasing amounts of data that have to be analysed by researchers. For example, astronomy has seen a shift from painstaking and time-consuming observations of a single star and its spectra to fibre-fed spectrographs that are able to simultaneously acquire spectra of hundreds to thousands of stars in a comparable time. Given the vast amounts of acquired data that can not in reasonable time be individually inspected and analysed, we are increasingly relying on computer algorithms to ease and speed-up those steps for us.

Among the studies dedicated to the exploration of stellar objects, we can choose spectroscopic observations among the following ongoing surveys: \G\ spacecraft \cite{2016A&A...595A...1G}, GALactic Archaeology with HERMES (GALAH \cite{2017MNRAS.465.3203M}), Apache Point Observatory Galactic Evolution Experiment (APOGEE \cite{2017AJ....154...94M}), and Large Sky Area Multi-Object Fibre Spectroscopic Telescope (LAMOST \cite{2012RAA....12.1197C}). In addition to them, we can also obtain data from the past large surveys Sloan Extension for Galactic Understanding and Exploration (SEGUE \cite{2009AJ....137.4377Y}), RAdial Velocity Experiment (RAVE \cite{2017AJ....153...75K}), and Gaia-ESO Survey (GES \cite{2012Msngr.147...25G}). The technological development is not stopping, bringing us even more complex telescopes and spectrographs that will acquire even more spectra in a single exposure. We, therefore, look forward to the observation start of surveys 4-metre Multi-Object Spectrograph Telescope (4MOST \cite{2012SPIE.8446E..0TD}), WHT Enhanced Area Velocity Explorer (WEAVE \cite{2012SPIE.8446E..0PD}), and Funnel Web.

All mentioned surveys differ in the target selection methodology (magnitude and/or colour thresholds), properties of observed spectra (resolution, wavelength coverage etc.), and sky coverage, but all produce normal and peculiar spectra that have to identified and analysed. Because of a large number of observations, it is not feasible or justified to manually reduce and analyse all those spectra, therefore automatic pipelines have to be used. The first step in understanding observed spectroscopic data is a reduction pipeline, tailored to a specific telescopic and spectrograph setup \cite{2017MNRAS.464.1259K, 2019arXiv191202905A, 2020arXiv200204377S}, that will homogeneously prepare all spectra for further use and potentially add quality flags, warning a user that something might be wrong.

After the data have been prepared, machine learning approaches have been used in many different ways to extract further information from the observed spectra. The most important stellar properties in galactic archaeology are iron content \Feh\ and individual chemical abundances. The first gives information about the amount of iron in the observable outer layers of the star compared to the amount of hydrogen. It is easily measurable as iron atoms cause numerous absorption lines in a spectrum. Similarly, individual abundances give the abundance of analysed element X compared to iron, usually denoted as \Xfe. Both of them are measured on the logarithm scale and compared to abundances found in the Sun. The iron abundance of our Sun is in the given system defined as \Feh~=~0.

When having access to a small set of observations with correctly determined parameters, different approaches have been used to project them onto the whole dataset. Currently, the most widely used approaches are \TC\ \cite{2015ApJ...808...16N, buder2018} and Payne \cite{2019ApJ...879...69T}, which fall into the category of generative approaches. Such an approach, in order to infer parameters of the analysed spectrum, internally generates a model spectrum that is compared to observations. Another large group of parameter determination machine learning approaches are regression algorithms that directly infer parameters defined by the training set. Of many existing regression approaches, currently, the most explored is the use of various neural network architectures \cite{2015MNRAS.452..158Y, 2019MNRAS.483.3255L, 2020ApJ...891...23W, 2020arXiv200208390O}. The usage ranges from deep fully connected networks and convolutional networks to autoencoder structures that first extract latent spectral features which are later used in the regression procedure. As both methodologies do not include any knowledge about stellar physics and are trained on a limited set of data, usually much smaller than all possibilities found in Galaxy, their results could also be misleading. Questionable results can be returned in the case of extrapolation when observed spectrum has parameters outside of a training grid or when an algorithm is used to train on an element that is not present in the used spectral range. Of course, an algorithm could find some average correlations of an unobservable feature with other features, but this has no underlying basis with observations or physics and could, therefore, give us wrong results.

Moving to chemical tagging experiments, that builds on top of previously determined stellar parameters and abundances, a shift from supervised to unsupervised machine learning algorithms is sensible as we rarely know the desired result of a conducted experiment. In the literature, chemical tagging experiments often rely on existing unsupervised clustering methods such as K-means \cite{2015A&A...577A..47B, 2016ApJ...833..262H, 2018ApJ...860...70C}, DBSCAN \cite{2019MNRAS.487..871P}, t-SNE \cite{2018MNRAS.473.4612K, 2018A&A...619A.125A}, hierarchical clustering \cite{2017MNRAS.467.1140J, 2018A&A...618A..65B}, and other custom methodologies \cite{2019ApJ...887...73C}. Results of those algorithms are not some numerical values, but groups of data points with similar input parameters, such as abundances in the case of chemical tagging.

For some other surveys, such as \G, users do not have full access to raw observations but have to trust to published stellar properties and their warning signs - such as uncertainties and quality flags - to filter out published data before performing any machine learning operations. Understanding of used data and proper treatment of warning flags of uncertain data is therefore essential before blindly applying any machine learning algorithm to an unknown data set. Failing to do so, we can get stuck in a process that is in computer science commonly referred to as "garbage in, garbage out". A phrase that emphases use of clean input training data as it is directly transferred upon investigated test set. Many attempts have been made to filter out \G\ kinematic and astrometric parameters effectively, but so far the most commonly used and recommended parameter is Renormalised Unit Weight Error (RUWE) \cite{ruwe} that is used in numerous explorations of the latest dataset. 

Various machine learning approaches have been so far applied to the latest \G\ DR2 dataset, among other trying to classify variable stars \cite{2020MNRAS.493.2981B}, determine stellar effective temperature \cite{2019AJ....158...93B}, catalogue young stellar objects \cite{2019MNRAS.487.2522M}, study streams and moving groups in the galactic halo \cite{2017A&A...598A..58H, 2020MNRAS.492.1370B}, identify accreated stars and structures \cite{2019arXiv190706652O, 2019arXiv190707681N}, track hypervelocity stars \cite{2017MNRAS.470.1388M}, delineate between stellar and extragalactic sources \cite{2018RAA....18..118B, 2019MNRAS.490.5615B}, determine interstellar extinction rates \cite{2020AJ....159...84B}, and define new open stellar clusters \cite{2020A&A...635A..45C}.

\section{Our exploration of observed data}
Our research into open stellar clusters and spectroscopically peculiar stars consists of three related subtopics that are further explained in the following chapters: analysis of possible runaway stars that were ejected from their birth clusters (based on \G\ kinematic and positional measurements and the GALAH spectroscopic parameters and abundances), identification of chemically peculiar and emission stellar spectra (based on the GALAH spectroscopic data), and investigation of spectroscopic solar twin stars and their multiplicity (based on multiple photometric surveys, \G\ distances, and the GALAH spectroscopic data).

Investigated subtopics share the observational data, but have vastly different physical significance with the common goal of understanding current chemical composition and structure of Galaxy. Open clusters that are thought to be chemically the most homogeneous structures can finally be analysed in dept to uncover if this is true. Additionally, we could find possible sources of intra chemical enrichment by evolved stars. Determination of precise chemical composition is demanding, especially when observing stars that do not behave as the majority of the population. As we want to clean the \Gh\ dataset, we focused on finding carbon enriched and emission stars that could endanger the chemical analysis. With limited observing time and capability, the whole sky can not be thoroughly observed by only one survey. Therefore we need to combine and equalise results from multiple surveys. Standard way for inter-calibration is using solar-like stars as Sun is the best observed and studied star with exactly known parameters and composition.

% clusters and ejected stars
\subsection{Open clusters and ejected stars}
To study ejected stars, we built upon available research that already defined membership possibilities and cluster parameters for more than 1000 open clusters in the Galaxy \cite{2018A&A...618A..93C} using the latest \G\ DR2 data. The refined membership probabilities of previously known open clusters \cite{2013A&A...558A..53K} were redefined using improved positional and kinematic information. To further define the best cluster members, we additionally used radial velocities to sift out outlying members and multiple stars. With the initial members of a cluster in place, we can continue with the analysis on ejected stars located far away from their cluster centre. Observed full 6D positional and kinematics vector for every star enables simulating their position in the Galaxy at different epochs. By integrating those properties, we can model the evolution of stellar clusters and their dissolution \cite{1998A&A...337..363P}. It has already been shown that it is possible to retrace some of the nearest runaway stars back to its original cluster using older Hipparcos astrometric observations \cite{2000ApJ...544L.133H}.

To perform a similar task, we used the latest \G\ astrometric measurements supplemented with radial velocities measured by the GALAH spectroscopic survey. Their observations were done for fainter sources that are unaccessible for the spectrograph onboard the spacecraft. We integrate the movement of stars inside and outside a cluster backwards in time to determine potential points in time where their orbits around the centre of the Galaxy intersect. Those intersections give us a list of candidates that were once members of a cluster.

The prime focus of this whole open cluster exploration was determination if machine learning approaches can be readily used for blind chemical tagging of known stellar structures. With this task in mind, we first determined homogeneity and trends of individual chemical abundances for clusters observed among the GALAH spectra. As every cluster is surrounded with numerous unrelated field stars, we explored if the chemical signature of a cluster is any different of its surrounding. More significant the difference among both chemical signatures, easier the chemical tagging is. In the case of very similar chemical compositions, kinematic information of stars gives us and additional information, and sometimes the only one, that can help with the delineation among field, cluster and ejected stars. An in-depth explanation of our work and results is given in Chapter \ref{chap:clusters}.

% peculiar stars
\subsection{Spectroscopically peculiar stars}
Every large, unbiased spectroscopic observational set is prone to target some peculiar stars whose spectrum does not resemble the majority of so-called normal spectra. One of the ways to produce those spectra is using simulations that try to reproduce the complete physics of stars' interior \cite{2008A&A...486..951G}. Given the complexity of those computations, different approximations and simplifications are made. As this might introduce unwanted differences towards typically observed spectra, we prefer to relay on observations itself to produce a set of most common normal-looking spectra. The GALAH and other similarly vast spectral sets have enough diversity and quantity of observations to produce normal-looking spectra in such manner.

During our initial search trough the GALAH spectral dataset \cite{2017ApJS..228...24T}, we confirmed that peculiar spectra could also be found in our acquired data. For a more consistent search of those peculiarities, we employed a multitude of supervised and unsupervised machine learning techniques to broaden the search onto a complete set of spectra. In it, we browsed for chemically peculiar spectra and spectra with pronounced emission lines. All of those special spectral types need to be identified as thoroughly as possible because they can be interesting for further studies on the one hand and might be difficult to correctly determine their physical parameters and abundances using automated pipelines on the other hand.

The supervised search for peculiar stars was based on the generation of normal reference spectra without any peculiarities that were compared with observed spectra. Any mismatch at the predetermined expected wavelengths was thoroughly measured and analysed in order to extract some physically meaningful explanations about the observed peculiarity. Reference spectra were in our case constructed using two different methodologies. During the supervised construction, we compared observed spectra towards the median spectrum of stars with similar physical properties. The more complicated unsupervised methodology performed spectrum generation using neural network autoencoder that extracts the most important latent features from a spectrum and reconstructs it using those few latent features. A result of this spectral compression and un-compression was a peculiarity free reference spectrum.

For the same task of classifying peculiar stars, we also applied unsupervised clustering technique t-SNE \cite{2013arXiv1301.3342V} to normalised acquired spectra. The algorithm treats individual spectra as vectors of multiple features of very high dimensionality whose complexity can not be perceived or visualised by humans. To convert this multitude of dimensions into a visually manageable form, the algorithm first computes similarities between all those vectors and groups spectra based on their similarity. The final result is a 2D or 3D map of points. In such a map, it is easy to select denser groups of data points and investigate if all selected spectra have a peculiarity that we were looking for. Further explanations of our search for peculiar spectra and results are given in Chapters \ref{chap:peculiars_chem} and \ref{chap:peculiars_emis}.

% solar twins and muliplities
\subsection{Spectroscopic solar twins and their multiplicity}
Among all, the best-studied chemically peculiar star is our own Sun. When compared to spectroscopically and/or photometrically similar stars, also named solar twins (as defined by \cite{2017AN....338..442A}), it shows signs of under-abundance of volatile chemical elements \cite{2009ApJ...704L..66M} that may hint to a formation of a solar system around the observed star. When searching for solar twins, we do not consider only the chemical composition of stars, but also their physical parameters. For us, it was interesting to see where in the Galaxy we could find solar twins observed by the GALAH. Especially interesting are solar twins embedded in open clusters for which we can study their possible differences in composition towards their birth open cluster. Studies revealed that old open cluster M67 (also observed in the GALAH) seems to contain several stars than could be regarded as solar twins \cite{2009MmSAI..80..125B, 2016MNRAS.463..696L}.

Solar twins are interesting, even for many other studies. From their observed luminosity and known luminosity of the Sun, we can accurately determine their distance. Underabundace of volatile chemical elements in solar twins can be correlated with the presence of planetary systems around selected stars. The fraction of studied stars in the GALAH survey had or will be studied by finished K2 \cite{2014PASP..126..398H} and future TESS \cite{2014SPIE.9143E..20R} spaceborne missions that are searching for planets around other stars. Matching known stars with planetary systems with our GALAH sample might reveal abundance patterns that can hint at the presence of rocky or gaseous planets.

To search for solar twins, we used raw spectra and compared them to reference solar spectrum, that was constructed by averaging a multitude of acquired twilight flats. The selection of the best candidates was based on similarity metrics, where we selected only a few percents of the best matching spectra. Close inspection of absolute magnitudes (observed magnitude corrected for stellar distance) showed that some of the stars in the selection looks too bright for their spectral type and distance. To investigate possible physical scenarios that would reproduce the observations, we build a spectroscopic and a photometric model of a single star. Combined light and spectrum of multiple single stars revealed that observed objects might contain multiple stellar sources that are slowly revolving around their common mass centre. Further details about the search for solar twins and their multiplicity modelling is given in Chapter \ref{chap:twins}.