%Razširjeni povzetek v slovenskem jeziku naj bo dolg vsaj 10 strani. 
%Vključuje naj tudi slike, tabele in enačbe, ki so nujne za razumevanje besedila povzetka.

% -----------------------------------------------------------------------
\section{Uvod}
\label{sec:slo_uvod}
Galaksija, kot jo lahko občudujemo dandanes na podlagi opazovanj ni nastala v enem trenutku, ampak je rezultat postopnega rojevanja zvezd in dodatnih zunanjih gravitacijskih vplivov. Njena najmlajši razpoznavni sestavni deli so razsute zvezne kopice, ki so nastale iz istega molekularnega oblaka snovi in zaradi tega vse zvezde v njej še vedno ohranjajo določene skupne lastnosti. Najočitnejše lastnosti so njihova zgoščena lokacija na nebu in skupna smer premikanja po Galaksiji. Poleg tega lahko opazujemo še kemično sestavo posameznih zvezd v kopici, ki nam podaja neposredno informacijo o sestavi prvotnega oblaka materiala. Kot vsaka gravitacijsko slabše povezana tvorba tudi razsute kopice sčasoma razpadejo in se pomešajo med okoliške zvezde, trajanje takega procesa pa je odvisen od njene začetne mase. Glavna mehanizma za izgubo zvezd kopice sta izmetavanje članov tekom bližnjih srečanj znotraj kopice ter počasno gravitacijsko odstranjevanje zaradi vpliva zvezd izven kopice. Gravitacijsko trenje in odstranjevanje je najučinkovitejše ko kopica potuje v gosto naseljenem področju.  Takrat zvezde kopice vlečejo navidezno stacionarne okoliške zvezde proti sebi. Posledično izgubijo nekaj energije, zvezdam kopice se zmanjša hitrost in posledično orbita potovanja okrog centra Galaksije. Takšne interakcije lahko trajajo dokler se kopica postopoma popolno ne zlije z okoliškimi strukturami.

Čeprav so kopice zaradi svojega izvora iz skupnega homogenega oblaka snovi idealne za raziskovanje strukture in gradnikov Galaksije, je njihova pričakovana življenjska doba približno 100~Myr, najgostejše strukture pa lahko preživijo tudi nekaj milijard let. Tako kratka življenjska doba nam skrajšuje čas v katerem so kopice na voljo za raziskovanje, vendar nam po drugi strani njihov hiter razvoj obenem ponuja možnost hkratnega opazovanja dinamično različno razvitih struktur.

\subsection{Razsute kopice v dobi satelita \G}
Najnovejši podatki s satelita \G\ so močno izboljšali znane podatke o zvezdah. Končno poznamo natančne informacije o svetlosti, poziciji, oddaljenosti in popolnem gibanju tako temnih kot svetlih zvezd po celotnem nebu. Takšen podatkovni set je pripomogel tudi k ponovnemu zanimanju za razsute kopice, kar v zadnjem času opazimo kot povečano število novih dognanj o teh strukturah.

\subsection{Metoda kemičnih podpisov}


\subsection{Posebni zvezdni spektri}


\subsection{Strojno učenje in obširni pregledi neba}


\subsection{Naše raziskave}


% -----------------------------------------------------------------------
\section{Pregledi neba}
\label{sec:slo_pregledi}
Astronomska opazovanja so se v zadnjih desetletjih močno spremenila, saj gradimo vedno večje teleskope in kompleksnejše sisteme, ki namesto posameznih namensko izbranih objektov hkrati opazujejo več sto ali tisoč objektov. Seveda takšen način opazovanj za seboj povleče tudi zahtevnejšo obdelavo zbranih surovih podatkov in drugačen pristop k znanosti, ki jo iz obdelanih podatkov lahko pridobimo. Avtomatski, ponavadi le deloma nadzorovani sistemi za obdelavo in analizo podatkov najpogosteje niso prilagojeni za celoten možen razpon opazovanj. Zato moramo biti uporabniki še bolj pozorni na razne opozorilne znake, ki nakazujejo na možne nepravilnosti tekom teh procedur in tudi poznati same procedure, saj se le tako lahko zavedamo vseh možnih pasti. Nekaj takšnih posebnosti pri razumevanju pridobljenih podatkov si bomo pogledali tudi v naslednjih poglavjih.

Izmed vseh možnih astronomskih podatkovnih setov, smo se v disertaciji osredotočil na sledeče tri preglede neba, ki nam podajajo informacije o svetlosti zvezd, njihovi sestavi, oddaljenosti, premikanju po Galaksiji in mnogo drugih parametrov, ki jih lahko razberemo iz opazovanj.

\subsection{Vesoljska misija \G}
\label{sec:slo_gaia}
Satelit \G\ je Evropska vesoljska agencija (Esa) izstrelila v drugo Langranevo točko sistema Zemlja-Luna, kjer že od julija 2014 neprestano skenira zvezdno nebo. Končni cilj misije je izdelava najbolj natančne zvezdne karte zvezd do magnitude $\sim$20,7 ter pridobitev točnih informacij o njihovi oddaljenosti in lastnem gibanju. Končni katalog bo tako vseboval več kot milijardo zvezde.

To bo satelit zmogel le z sistematičnem pregledovanjem neba, ki je sestavljeno iz dveh neodvisnost rotirajočih se gibanj: rotiranje okrog lasne osi vsakih šest ur ter počasna 63 dnevna precesija lasne osi, ki je glede na Sonce postavljena pod kotom 45$^\circ$. V prvih petih letih predvidenega delovanja je \G\ opravila 29 period gibanja smeri osi, kar je vodilo v optimalno pokritost zvezdnega neba. Vsaka zvezda je bila tako opazovana približno 70-krat za določitev njene pozicije in barve ter 40-krat za izmero radialne hitrost. V podaljšku načrtovane misije, ki se je začel julija 2019, je satelit nadaljeval svoje pregledovanje, vendar z obratno smerjo gibanja osi.

Teko rotacije satelita, se opazovane zvezde počasi premikajo čez njegovo goriščno ravnino, kjer je nameščenih 106 CCD detektorjev (prikaz na Sliki \ref{fig:gaia_ccd}). Vsaka zvezda, ki jo opazujemo, se tako zapelje čez kompleksen sistem detektorjev ki zaporedoma določijo njeno lokacijo na nebu, spektralno porazdelitev energije in posnamejo njen ozkopasovni spekter srednje ločljivosti.

\subsubsection{Fotometrija in astrometrija}
Ob prihodu v goriščno ravnino zbrana svetloba zvezde najprej posveti na detektor za določanje izvorov (angl. Sky Mapper - SM), ki samodejno zazna zvezde. Označi zvezde, ki so svetlejše od magnitude 20,7 in temnejše od magnitude 3 saj so preostale presvetle za njegovo delovanje. Po zaznavi se zvezda premakne na površinsko največji del CCD detektorjev imenovan astrometrično polje (angl. Astrometric Field - AF). Ta zabeleži natančno pozicijo, ki bo kasneje uporabna tudi za določitev lastne hitrosti in oddaljenosti, ter svetlost že prej detektiranih zvezd. Zabeležena svetlost zvezde nakazuje njeno sešteto svetlost v širokem spektralnem področju od 3300 do 10.500~\AA. Izbrano področje imenujemo tudi kot \G\ G fotometrični pas.

Zadnjo meritev svetlosti nato opravi še spektrofotometrični sistem, ki z disperzijskim detektorjem izmeri natančne svetlobne tokove v večih ozkopasovnih pod-pasovih prej omenjenega širokega \G\ G pasu. Modri fotometer (angl. Blue Photometer - BP) posname spekter nizke ločljivosti vseh zvezd v območju od 3300 do 6800~\AA. Integrirano magnitudo označimo kot BP ali G$_{BP}$. Podobno rdeči fotometer (angl. Red Photometer - RP) analizira svetlobo v območju od 6300 do 10.500~\AA, kjer je njena integrirana vrednost podajama kot RP ali G$_{RP}$ magnituda. Oba nizko ločljivost spektrometra BP in RP v svojem danem območju posnameta spekter v 12 pod-območjih. Te meritve uporabnikom še niso na voljo in bodo predvidoma objavljene kot del tretjega izida \G\ podatkov.

\subsubsection{Spektroskopija}
Končna meritev, ki jo \G\ opravi je spektroskopija celotnega opazovanega zvezdnega polja. Vgrajeni spektrometer za radialne hitrost (angl. Radial Velocity Spectrometer - RVS) za zvezde svetlejše od 16. magnitude zbere spektre srednje ločljivosti z resolucijsko močjo R $\sim$11.700 na spektralnem območju od 8450 do 8720~\AA. Področje pokriva območje treh enkrat ioniziranih kalcijevih absorpcijskih črt, ki so primerne za določevanje radialnih hitrost raznovrstnih zvezd na širokem temperaturnem območju.

\subsubsection{\G\ DR2}
\rb{Morda na hitro kaj o tem, odvisno od že zapolnjenih strani}.

\subsection{Pregled neba \Gh}
\label{sec:slo_galah}
Večinski del podatkov uporabljenih za izdelavo te disertacije je bil pridobljen iz arhiva še vedno tekočega pregleda The GALactic Archaeology with HERMES (\Gh, \cite{2015MNRAS.449.2604D}) in njemu sorodnih pregledov, ki so bili posneti s spektrografom High Efficiency and Resolution Multi-Element Spectrograph (HERMES, \cite{2010SPIE.7735E..09B, 2015JATIS...1c5002S}), ki omogoča hkratno opazovanje skoraj 400 zvezd. Spektrograf je za zbiranje svetlobe nameščen na štiri-metrskem Anglo-avstralskem teleskopu (AAT) v observatoriju Siding Springs, Avstralija. Primarni cilj pregleda \Gh\ je razumevanje procesov, ki so vodili do današnje strukture Galaksije, preko natančnih meritev kemičnih zastopanosti zvezd različnih komponent Galaksije. Resolucijska moč sistema je $\sim$28.000 in pokriva štiri spektralna področja (4713 -- 4903~\AA, 5648 -- 5873~\AA, 6478 -- 6737~\AA in 7585 -- 7887~\AA), ki so neodvisno zajeta in obdelana v štirih ločenih delih spektrografa (za shematičen prikaz glejte Sliko \ref{fig:hermes_2df}). To področja poimenujemo tudi kot modri, zeleni, rdeči in bližnje infrardeči spektralni del. Skupaj pokrivajo približno 1000~\AA\ in vsebujejo pomembni vodikovi absorpcijski črti H$\alpha$ in H$\beta$. Selekcijska funkcija za izbiro tarč je za svetla polja omejena na zvezde z magnitudo 10<V<12 in temna polja z magnitudo 12<V<14. Razen izogibanja delom s preveliko gostoto možnih zvezd okrog Galaktične ravnine ($|b|$~>~$10^\circ$) ni izvedene nobene dodatne vnaprejšnje izbire zvezd.

Vsa hkratna opazovanja zvezd so zbrana na dvodimenzionalni sliki, ki jo je potrebno obdelati, da z nje pridobimo množico enodimenzionalnih zvezdnih spektrov. Avtomatska procedura ze redukcijo opazovanj je podrobno opisana in razložena v \citet{2017MNRAS.464.1259K}. Po obdelavi iz spektrov pridobimo radialno hitrost zvezde, njene osnovne fizikalne lastnosti in zastopanosti kemičnih elementov na površju zvezde. Tekom opazovanj se je kompleksnost algoritmov za pridobitev teh parametrov stopnjevala in v aktualni tretji objavi podatkov vključuje tudi lastnosti, ki jih je določila \G. 

\subsection{Asiago}
\label{sec:slo_asiago}
Tekom študija sem se večkrat odpravil tudi na observatorij Asiago v Italiji, ki ponuja zelo drugačen način opazovanja kot prej omenjena masivna pregleda neba. Uporabljen 1,82~m teleskop Copernico je v času naših obiskov okrog polne Lune prilagojen za spektroskopska opazovanja posameznih zanimivih objektov. Zaradi svojega načina delovanja je sistem primeren za podrobnejšo oziroma časovno zamaknjeno analizo spektrov zanimiv zvezd, ki smo jih prej identificirali v drugih obširnih pregledih neba. Teleskop je iz nadzorne sobe voden preko oddaljene povezave saj se nahaja na bližnji vzpetini Ekar na nadmorski višini 1.366~m. 

Nameščen spektrograf tipa Echelle omogoča zajem spektrov z resolucijsko močjo R $\sim$20.000 na širokem spektralnem področju med 3600 in 7400~\AA. Zajet spekter je razdeljen na 30 interferenčnih redov, ki se medsebojno delno prekrivajo kar omogoča izdelavo neprekinjenega spektra zvezde na tem področju. Sistem omogoča zajem spektrov z visokim razmerjem med signalom in šumom za zvezde z magnitudo V < 10. 

Njegova lokacija na nasprotni zemljini polobli kot teleskop AAT zmanjšuje presek z opazovanimi zvezdami pregleda GALAH. Kljub temu smo zbrane podatke podatke uspešno uporabili kot dodatek k analizam v Poglavjih \ref{chap:peculiars_chem} in \ref{chap:twins} ter kot glavni del objavljenega znanstvenega članka \cite{2019MNRAS.488.5536M} in astronomskega telegrama \cite{2019ATel13340....1M}.

% -----------------------------------------------------------------------
\section{Kemično in dinamično raziskovanje razsutih kopic}
\label{sec:slo_kopice_taziskovanje}


% -----------------------------------------------------------------------
\section{Kemično posebne zvezde}
\label{sec:slo_c_peculiars}


% -----------------------------------------------------------------------
\section{Emisijske zvezde}
\label{sec:slo_emisijske}


% -----------------------------------------------------------------------
\section{Soncu podobne večkratne zvezde}
\label{sec:slo_soncevi_veckratniki}


% -----------------------------------------------------------------------
\section{Zaključki in prihodnje študije}
\label{sec:slo_zakljucek}
