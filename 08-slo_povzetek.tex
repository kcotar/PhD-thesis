%Razširjeni povzetek v slovenskem jeziku naj bo dolg vsaj 10 strani. 
%Vključuje naj tudi slike, tabele in enačbe, ki so nujne za razumevanje besedila povzetka.

% -----------------------------------------------------------------------
\section{Uvod}
\label{sec:slo_uvod}


\subsection{Razsute kopice v dobi satelita \G}

\subsection{Metoda kemičnih podpisov}

\subsection{Posebni zvezdni spektri}

\subsection{Strojno učenje in obširni pregledi neba}

\subsection{Naše raziskave}

% -----------------------------------------------------------------------
\section{Pregledi neba}
\label{sec:slo_pregledi}
Astronomska opazovanja so se v zadnjih desetletjih močno spremenila, saj gradimo vedno večje teleskope in kompleksnejše sisteme, ki namesto posameznih namensko izbranih objektov hkrati opazujejo več sto ali tisoč objektov. Seveda takšen način opazovanj za seboj povleče tudi zahtevnejšo obdelavo zbranih surovih podatkov in drugačen pristop k znanosti, ki jo iz obdelanih podatkov lahko pridobimo. Avtomatski, ponavadi le deloma nadzorovani sistemi za obdelavo in analizo podatkov najpogosteje niso prilagojeni za celoten možen razpon opazovanj. Zato moramo biti uporabniki še bolj pozorni na razne opozorilne znake, ki nakazujejo na možne nepravilnosti tekom teh procedur in tudi poznati same procedure, saj se le tako lahko zavedamo vseh možnih pasti. Nekaj takšnih posebnosti pri razumevanju pridobljenih podatkov si bomo pogledali tudi v naslednjih poglavjih.

Izmed vseh možnih astronomskih podatkovnih setov, smo se v disertaciji osredotočil na sledeče tri preglede neba, ki nam podajajo informacije of svetlosti zvezd, njihovi sestavi, oddaljenosti, premikanju po vesolju in mnogo drugih parametrov, ki jih lahko razberemo iz opazovanj.

\subsection{Vesoljska misija \G}

\subsection{Pregled neba GALAH}

\subsection{Asiago}


% -----------------------------------------------------------------------
\section{Kemično in dinamično raziskovanje razsutih kopic}
\label{sec:slo_kopice_taziskovanje}


% -----------------------------------------------------------------------
\section{Kemično posebne zvezde}
\label{sec:slo_c_peculiars}


% -----------------------------------------------------------------------
\section{Emisijske zvezde}
\label{sec:slo_emisijske}


% -----------------------------------------------------------------------
\section{Soncu podobne večkratne zvezde}
\label{sec:slo_soncevi_veckratniki}


% -----------------------------------------------------------------------
\section{Zaključki in prihodnje študije}
\label{sec:slo_zakljucek}
