%Razširjeni povzetek v slovenskem jeziku naj bo dolg vsaj 10 strani. 
%Vključuje naj tudi slike, tabele in enačbe, ki so nujne za razumevanje besedila povzetka.

% -----------------------------------------------------------------------
% Slovenski daljši uvod s pregledom tematik in obdelanih področij
% -----------------------------------------------------------------------
\section{Uvod}
\label{sec:slo_uvod}
Galaksija, kot jo lahko občudujemo dandanes na podlagi opazovanj ni nastala v enem trenutku, ampak je rezultat postopnega rojevanja zvezd in dodatnih zunanjih gravitacijskih vplivov. Njeni najmlajši razpoznavni sestavni deli so razsute zvezdne kopice, ki so nastale iz istega molekularnega oblaka snovi in zaradi tega vse zvezde v njej še vedno ohranjajo določene skupne lastnosti. Najočitnejše lastnosti so njihova zgoščena lokacija na nebu in skupna smer premikanja po Galaksiji. Poleg tega lahko opazujemo še kemično sestavo posameznih zvezd v kopici, ki nam podaja neposredno informacijo o sestavi prvotnega oblaka materiala. Kot vsaka gravitacijsko slabše povezana tvorba tudi razsute kopice sčasoma razpadejo in se pomešajo med okoliške zvezde, trajanje takega procesa pa je odvisen od njene začetne mase. Glavna mehanizma za izgubo zvezd kopice sta izmetavanje članov tekom bližnjih srečanj znotraj kopice ter počasno gravitacijsko odstranjevanje zaradi vpliva zvezd izven kopice. Gravitacijsko trenje in odstranjevanje je najučinkovitejše, ko kopica potuje v gosto naseljenem področju. Takrat zvezde v kopici vlečejo navidezno stacionarne okoliške zvezde proti sebi. Posledično izgubijo nekaj energije, zvezdam kopice se zmanjša hitrost in posledično se spremeni tudi orbita potovanja okrog centra Galaksije. Takšne interakcije lahko trajajo dokler se kopica postopoma popolno ne zlije z okoliškimi strukturami.

Čeprav so kopice zaradi svojega izvora iz skupnega homogenega oblaka snovi idealne za raziskovanje strukture in gradnikov Galaksije, je njihova pričakovana življenjska doba približno 100 milijonov let. Najgostejše strukture lahko preživijo tudi nekaj milijard let. Tako kratka življenjska doba nam skrajšuje čas v katerem so kopice na voljo za raziskovanje, vendar nam po drugi strani njihov hiter razvoj obenem ponuja možnost hkratnega opazovanja dinamično različno razvitih struktur.

\subsection{Razsute kopice v dobi satelita \G}
Najnovejši podatki s satelita \G\ so močno izboljšali znane podatke o zvezdah. Končno poznamo natančne informacije o svetlosti, poziciji, oddaljenosti in popolnem galaktičnem gibanju tako temnih kot svetlih zvezd po celotnem nebu. Takšen podatkovni set je pripomogel tudi k ponovnemu zanimanju za razsute kopice, kar v zadnjem času opazimo kot povečano število novih dognanj o teh strukturah.

Prvi korak v raziskovanju kopic je natančna določitev članov posamezne kopice. Prvi postopki so temeljili na preprostem izbiranju zvezd na podlagi njihove lokacije na nebu, saj kopico opazimo kot lokalno zgoščitev števila zvezd na majhnem območju. Dandanes za tako določevanje uporabljamo kompleksnejše algoritme, saj lahko uporabimo mnogo dodatnih informacij poleg pozicije zvezd. Najpogostejši način izbiranja temelji na določitve zgoščine v kinematičnem prostoru, saj ima večina kopic distinktno drugačno lastno in radialno gibanje v primerjavi z okolico. S poznavanjem barve, navidezne svetlosti zvezde in njene oddaljenosti lahko preko teoretičnega modela razvoja zvezd na HR diagramu izločimo še preostale napačno izbrane člane in posledično določimo natančno starost kopice.

Čeprav so napake meritev paralakse za bolj oddaljene kopice primerljive z njihovo velikostjo, vseeno lahko določimo njihovo približno obliko. Zaradi merskih napak le-ta ni pričakovane sferične oblike, ampak je razpotegnjena v radialni smeri proč od opazovalca. Poznavanje oblike in kopice in razporeditve članov nam omogoča študije notranjih dinamičnih proces in porazdelitev mase zvezd. Še bolj natančne študije dogajanja znotraj samih kopic bodo omogočene z naslednjim, tretjim izidom podatkov \G, za katerega pričakujemo še dodatno zmanjšanje merskih nedoločenosti.

\subsection{Metoda kemičnih podpisov}
Vse prej opisane metodologije potrebujejo le popolne podatke o lokaciji in gibanju zvezd. Napredek opazovalnih metod in obdelave podatkov nam v zadnjem času omogoča, da presežemo te postopke in v analize vključimo še mnogo dimenzionalne podatke o kemični sestavi zvezd - postopek poznan pod imenom metoda kemičnih podpisov \cite{2002ARA&A..40..487F, 2010ApJ...721..582B}. Ta metoda nam omogoča grupiranje zvezd na njihove izvorne oblake snovi le na podlagi njihovih kemičnih podpisov in s tem odpira vrata v področje podrobnega raziskovanja formiranja in evolucije Galaksije. Takšno popolnoma slepo kemično določanje za razsute kopice trenutno še ni bilo pokazano, razen v redkih primerih, ko je kopica zašla v drug del Galaksije in tako imela močno drugačen kemični podpis v primerjavi z okoliškimi zvezdami \cite{2016ApJ...833..262H}. Večji uspeh ima metoda na razločevanju posameznih komponent Galaksije (disk, osrednja odebelitev in halo), saj so bile te formirane ob zelo različnih časih, ko je bila njihova izvorna snov različno obogatena s težjimi snovi. Te elemente, težje od helija, formirajo in oddajo v okoliški prostor razvite zvezde na koncu svojega življenjskega cikla.

\subsection{Posebni zvezdni spektri}
Metoda kemičnih podpisov ni omejena le z fizikalnim dogajanjem v Galaksiji in kvaliteto opazovalnih podatkov, ampak tudi s tipom zvezd, ki jih z njo analiziramo. Poleg množice normalnih tipov zvezd, katerih znamo opazovan spekter fizikalno rutinsko opisati, poznamo še posebne tipe zvezd, katerih spekter vsebuje nepričakovane in s tem hkrati tudi zelo zanimive spektralne posebnosti. Med posebne tipe zvezd, odvisno od fokusa raziskave, štejemo dvojne ali večkratne zvezde, zelo mlade zvezde, zvezde, ki vsebujejo nadpovprečne količine posameznega kemičnega elementa, zvezde s kompleksnimi emisijskimi profili in nenazadnje možne napake v obdelavi podatkov. Zaradi tega želimo takšne zvezde identificirati in jih izločiti iz standardnega nabora podatkov. Pri njihovi analizi lahko z avtomatskimi procedurami, ki pričakujejo normalne spektre, pridobimo potencialno napačne parametre, kar bi še dodatno oteževalo kemično določevanje skupkov sorodnih zvezd.

V vsakem obširnem pregledu neba, kot je \Gh, želimo zato pridobiti čimbolj podroben in raznolik seznam posebnih tipov zvezd, ki smo jih opazovali pri ne selektivnem izbiranju zvezd le po njihovi navidezni svetlosti. Njihova svetlost o sami zvezdi ne pove nič drugega kot to, ali bomo tekom opazovanja pridobili ustrezno kvaliteten spekter za nadaljno obdelavo.

\subsection{Strojno učenje in obširni pregledi neba}
V zadnjem desetletju smo priča porastu zbranih podatkov v večini znanstvenih področij, vključno z astronomijo. Tako velike količine zbranih podatkov je potrebno tudi analizirati, saj nam le varno shranjene na podatkovnih medijih ne prinašajo dodatne vrednosti. Na srečo vzporedno opazujemo tudi porast algoritmov strojnega učenja, ki nam pomagajo in omogočajo enostavnejše ter hitrejše razumevanje zbranih podatkov. Računalnik lahko za nas postori velik del opravil, ki so jih pred tem raziskovalci morali opravljati ročno. Tu pa lahko opazimo tudi slabost takšnih metod, saj le-te ponavadi ne vsebujejo dodatnega podrobnega človeškega fizikalnega znanja, ki bi jim pomagal v nepoznanih primerih. Da se izognemo takim nevšečnostim, je potrebno pred slepo uporabo metod strojnega učenja pripraviti smiselne učne primere, ki bodo algoritmu pomagali pravilno razumeti zadani problem.

V astronomski literaturi so bile nedavno metode strojnega učenja že uspešno uporabljene za določanje zvezdnih parametrov iz spektrov \cite{2015ApJ...808...16N, buder2018, 2019ApJ...879...69T}, razpoznavanje posebnih tipov zvezd \cite{2017ApJS..228...24T}, aplikacijo metode kemičnih podpisov \cite{2015A&A...577A..47B, 2016ApJ...833..262H,2018MNRAS.473.4612K, 2018A&A...619A.125A, 2017MNRAS.467.1140J, 2018A&A...618A..65B} in mnogo drugih primerov, primarno uporabljenih pri razvrščanju zvezdnih spektrov. Med vsemi možnimi tipi metod strojnega učenja pri obdelavi astronomskih spektralnih podatkovnih zbirk v zadnjem času opažamo porast zanimanja za uporabo različnih arhitektur nevronskih mrež \cite{2015MNRAS.452..158Y, 2019MNRAS.483.3255L, 2020ApJ...891...23W, 2020arXiv200208390O}.

Poleg spektroskopskih podatkov je zelo primeren vir podatkov za uporabo metod strojnega učenja tudi najaktualnejši izid \G\ podatkov, saj se v njem nahaja več kot milijarda zvezd. Njihova opazovanja so bila med drugim uspešno uporabljena za klasifikacijo tipov spremenljivih zvezd \cite{2020MNRAS.493.2981B}, določanje efektivne temperature zvezd \cite{2019AJ....158...93B}, določanje potokov zvezd v galaktičnem haloju \cite{2017A&A...598A..58H, 2020MNRAS.492.1370B}, identifikacija akrecijskih zvezd \cite{2019arXiv190706652O, 2019arXiv190707681N}, sledenje zelo hitrih zvezd \cite{2017MNRAS.470.1388M}, razločevanje med zvezdnimi izviri svetlobe v naši galaksiji in galaktičnimi izvori izven nje \cite{2018RAA....18..118B, 2019MNRAS.490.5615B}, izračun pordečitve zaradi medzvezdnega prahu \cite{2020AJ....159...84B} in razpoznavanje še neodkritih razsutih zvezdnih kopic \cite{2020A&A...635A..45C}.

\subsection{Naše raziskave}
\label{sec:slo_raziskave}
Naše raziskovanje primarno spektroskopskih podatkov razsutih zvezdnih kopic in posebnih tipov zvezd sestavljajo tri medsebojno povezane teme, ki so v nadaljnjih poglavjih tudi podrobneje predstavljene:

\begin{itemize}
	\item Pri raziskovanju razsutih kopic smo si v Poglavju \ref{chap:clusters} pomagali s podatki satelita \G, preko katerih smo kopico in njeno okolico razdelili na več komponent. Med njimi so nas najbolje zanimale zvezde, ki bi lahko v preteklosti bile izvržene iz same kopice daleč proč od nje v območje drugih zvezd. Na identificiranih primerih smo s pomočjo metode kemičnih podpisov preizkusili ali njene predpostavke držijo za naše podatke in ali bi uporabljeno metodo lahko aplicirali tudi za neznane strukture v Galaksiji.
	
	\item V podatkih spektroskopskega pregleda neba \Gh\ se poleg normalnih zvezd nahaja tudi kopica posebnih tipov zvezd, ki jih je potrebno čim učinkoviteje identificirati za natančnejše delovanje metode kemičnih podpisov. Za ta namen smo uporabili več nenadzorovanih in delno nadzorovanih metod strojnega učenja, ki so bile naučene na normaliziranih spektralnih podatkih. Tako smo med pregledom spektrov odkrili 918 zvezd z močno izraženimi molekularnimi ogljikovimi črtami, ki nakazujejo na nadpovprečno vsebnost ogljika v atmosferi zvezde. Odkrili smo tudi 10.364 spektrov z izraženimi vodikovimi emisijskimi črtami, ki so posledica različnih fizikalnih procesov v okolici zvezde. 
		
	\item Zanimali so nas tudi spektralni sončevi dvojniki, saj je Sonce v astronomiji uporabno kot standardna zvezda za raznorazne kalibracije. S primerjavo spektrov smo določili 329 opazovanih GALAH spektrov, ki so najbolj podobni Soncu in pregledali njihove fizikalne parametre ter absolutni izsev. Nekatere od njih imajo mnogo močnejši izsev od Sonca, kar nakazuje na prisotnost več podobnih zvezd v opazovanem sistemu. Možnost obstoja dvojnih in trojnih zvezd, ki imitirajo spekter Sonca, smo potrdili s simulacijami in združevanjem njihovih opazovalnih podatkov.
	
\end{itemize}

% -----------------------------------------------------------------------
% Kratki opisi uporabljenih podatkovnih setov
% -----------------------------------------------------------------------
\section{Pregledi neba}
\label{sec:slo_pregledi}
Astronomska opazovanja so se v zadnjih desetletjih močno spremenila, saj gradimo vedno večje teleskope in kompleksnejše sisteme, ki namesto posameznih namensko izbranih objektov hkrati opazujejo več sto ali tisoč objektov. Seveda takšen način opazovanj za seboj povleče tudi zahtevnejšo obdelavo zbranih surovih podatkov in drugačen pristop k znanosti, ki jo iz obdelanih podatkov lahko pridobimo. Avtomatski, ponavadi le deloma nadzorovani sistemi za obdelavo in analizo podatkov najpogosteje niso prilagojeni za celoten možen razpon opazovanj. Zato moramo biti uporabniki še bolj pozorni na razne opozorilne zastavice, ki nakazujejo na možne nepravilnosti tekom teh procedur in tudi poznati same procedure. Le tako se lahko zavedamo vseh možnih pasti. Nekaj takšnih posebnosti pri razumevanju pridobljenih podatkov si bomo pogledali tudi v naslednjih poglavjih.

Izmed vseh možnih astronomskih podatkovnih setov, smo se v disertaciji osredotočil na sledeče tri preglede neba. Ti nam podajajo informacije o svetlosti zvezd, njihovi sestavi, oddaljenosti, premikanju po Galaksiji in mnogo drugih parametrov, ki jih lahko razberemo iz opazovanj.

\subsection{Vesoljska misija \G}
\label{sec:slo_gaia}
Satelit \G\ \cite{2016A&A...595A...1G} je Evropska vesoljska agencija (Esa) izstrelila v drugo Langranevo točko sistema Zemlja-Luna, kjer že od julija 2014 neprestano skenira zvezdno nebo. Končni cilj misije je izdelava najbolj natančne zvezdne karte zvezd do magnitude $\sim$20,7 ter pridobitev točnih informacij o njihovi oddaljenosti in lastnem gibanju. Končni katalog bo tako vseboval več kot milijardo zvezde. To bo satelit zmogel le z sistematičnem pregledovanjem neba, ki je sestavljeno iz dveh neodvisnost rotirajočih se gibanj: rotiranje okrog lasne osi vsakih šest ur ter počasna 63 dnevna precesija lasne osi, ki je glede na Sonce postavljena pod kotom 45$^\circ$. V prvih petih letih predvidenega delovanja je \G\ opravila 29 period gibanja smeri osi, kar je vodilo v optimalno pokritost zvezdnega neba. Vsaka zvezda je bila tako opazovana približno 70-krat za določitev njene pozicije in barve ter 40-krat za izmero radialne hitrost. V podaljšku načrtovane misije, ki se je začel julija 2019, je satelit nadaljeval svoje pregledovanje, vendar z obratno smerjo gibanja osi.

Tekom rotacije satelita se opazovane zvezde počasi premikajo čez njegovo goriščno ravnino, kjer je nameščenih 106 CCD detektorjev (prikaz na Sliki \ref{fig:gaia_ccd}). Vsaka zvezda, ki jo opazujemo, se tako zapelje čez kompleksen sistem detektorjev ki zaporedoma določijo njeno lokacijo na nebu, spektralno porazdelitev energije in posnamejo njen ozkopasovni spekter srednje ločljivosti.

\subsubsection{Fotometrija in astrometrija}
Ob prihodu v goriščno ravnino zbrana svetloba zvezde najprej posveti na detektor za določanje izvorov (angl. Sky Mapper - SM), ki samodejno zazna zvezde. Označi zvezde, ki so svetlejše od magnitude 20,7 in temnejše od 3. magnitude saj so preostale presvetle za njegovo delovanje. Po zaznavi se zvezda premakne na površinsko največji del CCD detektorjev imenovan astrometrično polje (angl. Astrometric Field - AF). Ta zabeleži natančno pozicijo, ki bo kasneje uporabna tudi za določitev lastne hitrosti in oddaljenosti, ter svetlost že prej detektiranih zvezd. Zabeležena svetlost zvezde nakazuje njeno sešteto svetlost v širokem spektralnem področju od 3300 do 10.500~\AA. Izbrano območje je poimenovano \G\ G fotometrični pas.

Tekom obdelave pozicij zvezd na astrometričnem polju s prilagajanje ustrezne funkcije, pridobimo še podatke o paralaksi zvezde in njeno lastno gibanje. Če je izvor svetlobe sestavljen iz več neenakomerno svetlih komponent, se bo fotometričen center počasi premikal ob njihovem kroženju. Posledično to vpliva tudi na kvaliteto prilagojene funkcije. Za zaznavo takšnih astrometričnih večkratnih zvezd, bodo v prihodnosti pri analizi ustrezno prilagodili prilagajočo funkcijo in tako iz podatkov izluščili še možnost večkratnosti izvora in orbitalno periodo.

Zadnjo meritev svetlosti nato opravi še spektrofotometrični sistem \cite{2018arXiv180409368E}, ki z disperzijskim detektorjem izmeri natančne svetlobne tokove v večih ozkopasovnih pod-pasovih prej omenjenega širokega \G\ G pasu. Modri fotometer (angl. Blue Photometer - BP) posname spekter nizke ločljivosti vseh zvezd v območju od 3300 do 6800~\AA. Integrirano magnitudo označimo kot BP ali G$_{BP}$. Podobno rdeči fotometer (angl. Red Photometer - RP) analizira svetlobo v območju od 6300 do 10.500~\AA, kjer je njena integrirana vrednost podajama kot RP ali G$_{RP}$ magnituda. Oba nizko ločljivostna spektrometra BP in RP v svojem danem območju posnameta spekter v 12 pod-območjih. Te meritve uporabnikom še niso na voljo in bodo predvidoma objavljene kot del tretjega izida \G\ podatkov.

\subsubsection{Spektroskopija}
Končna meritev, ki jo \G\ opravi, je spektroskopija celotnega opazovanega zvezdnega polja. Vgrajeni spektrometer za radialne hitrost (angl. Radial Velocity Spectrometer - RVS, \cite{2018A&A...616A...5C}) za zvezde, svetlejše od 16. magnitude, zbere spektre srednje ločljivosti z resolucijsko močjo R $\sim$11.700 na spektralnem območju od 8450 do 8720~\AA. Razpon pokriva območja treh enkrat ioniziranih kalcijevih absorpcijskih črt, ki so primerne za določevanje radialnih hitrost raznovrstnih zvezd na širokem temperaturnem območju. Hitro potovanje zvezd po polju CCD detektorjev onemogoča dolgotrajno zbiranje svetlobe in s tem kvalitetnejše spektre. Za njihovo podrobnejšo analizo, poleg meritve radialne hitrosti, bodo pred končnim izidom vsa pridobljena opazovanja združena v en kvalitetnejši spekter. Podobno bodo v naslednjih izdajah podatkov poleg povprečne radialne hitrosti čez celotno obdobje opazovanj objavljene še hitrosti za vsako posamezno opazovanje, kar bo omogočilo študije velike števila dvojnih zvezd.

%\subsubsection{\G\ DR2}
%\rb{Morda na hitro kaj o tem, odvisno od že zapolnjenih strani}.

\subsection{Pregled neba \Gh}
\label{sec:slo_galah}
Večinski del podatkov, uporabljenih za izdelavo te disertacije, je bil pridobljen iz arhiva še vedno tekočega pregleda The GALactic Archaeology with HERMES (\Gh, \cite{2015MNRAS.449.2604D}) in njemu sorodnih pregledov, ki so bili posneti s spektrografom High Efficiency and Resolution Multi-Element Spectrograph (HERMES, \cite{2010SPIE.7735E..09B, 2015JATIS...1c5002S}), ki omogoča hkratno opazovanje skoraj 400 zvezd. Spektrograf je za zbiranje svetlobe nameščen na štiri-metrskem Anglo-avstralskem teleskopu (AAT) v observatoriju Siding Springs, Avstralija. Primarni cilj pregleda \Gh\ je razumevanje procesov, ki so vodili do današnje strukture Galaksije, preko natančnih meritev kemičnih zastopanosti zvezd različnih komponent Galaksije. Resolucijska moč sistema je $\sim$28.000 in pokriva štiri spektralna področja (4713 -- 4903~\AA, 5648 -- 5873~\AA, 6478 -- 6737~\AA\ in 7585 -- 7887~\AA), ki so neodvisno zajeta in obdelana v štirih ločenih delih spektrografa (za shematičen prikaz glejte Sliko \ref{fig:hermes_2df}). To področja poimenujemo tudi kot modri, zeleni, rdeči in bližnje infrardeči spektralni del. Skupaj pokrivajo približno 1000~\AA\ in vsebujejo pomembni vodikovi absorpcijski črti H$\alpha$ in H$\beta$. Selekcijska funkcija za izbiro tarč je za svetla polja omejena na zvezde z magnitudo 10<V<12 in temna polja z magnitudo 12<V<14. Razen izogibanja delom s preveliko gostoto možnih zvezd okrog Galaktične ravnine ($|b|$~>~$10^\circ$) ni izvedene nobene dodatne vnaprejšnje izbire zvezd.

Vsa hkratna opazovanja zvezd so zbrana na dvodimenzionalni sliki, ki jo je potrebno obdelati, da z nje pridobimo množico enodimenzionalnih zvezdnih spektrov. Avtomatska procedura za redukcijo opazovanj je podrobno opisana in razložena v \citet{2017MNRAS.464.1259K}. Po obdelavi iz spektrov pridobimo radialno hitrost zvezde, njene osnovne fizikalne lastnosti in zastopanosti kemičnih elementov na površju zvezde. Tekom opazovanj se je kompleksnost algoritmov za pridobitev teh parametrov stopnjevala in v aktualni tretji objavi podatkov vključuje tudi lastnosti, ki jih je določila \G. Vključitev dodatnih informacij nam pomaga pri boljši določitvi parametra \Logg, ki je slabo razpoznan iz izključno spektroskopskih meritev.

\subsection{Asiago}
\label{sec:slo_asiago}
Tekom študija sem se večkrat odpravil tudi na observatorij Asiago v Italiji, ki ponuja zelo drugačen način opazovanja kot prej omenjena masivna pregleda neba. Uporabljen 1,82~m teleskop Copernico je v času naših obiskov okrog polne Lune prilagojen za spektroskopska opazovanja posameznih zanimivih objektov. Zaradi svojega načina delovanja je sistem primeren za podrobnejšo oziroma časovno zamaknjeno analizo spektrov zanimiv zvezd, ki smo jih prej identificirali v drugih obširnih pregledih neba. Teleskop je iz nadzorne sobe voden preko oddaljene povezave saj se nahaja na bližnji vzpetini Ekar na nadmorski višini 1.366~m. 

Nameščen spektrograf tipa Echelle omogoča zajem spektrov z resolucijsko močjo R $\sim$20.000 na širokem spektralnem področju med 3600 in 7400~\AA. Zajet spekter je razdeljen na 30 interferenčnih redov, ki se medsebojno delno prekrivajo. To omogoča izdelavo neprekinjenega spektra zvezde na tem spektralnem področju. Sistem omogoča zajem spektrov z visokim razmerjem med signalom in šumom za zvezde z magnitudo V~<~10. 

Njegova lokacija na nasprotni zemljini polobli kot teleskop AAT zmanjšuje presek z opazovanimi zvezdami pregleda GALAH. Kljub temu smo zbrane podatke uspešno uporabili kot dodatek k analizam v Poglavjih \ref{chap:peculiars_chem} in \ref{chap:twins}, ter kot glavni del objavljenega znanstvenega članka \cite{2019MNRAS.488.5536M} in astronomskega telegrama \cite{2019ATel13340....1M}.

% -----------------------------------------------------------------------
% Opisi posameznih obdelanih tematik
% -----------------------------------------------------------------------
\section{Kemično in dinamično raziskovanje razsutih kopic}
\label{sec:slo_kopice_taziskovanje}
Med zvezdnimi polji, opazovanimi tekom pregleda \Gh, najdemo tudi člane razsutih zvezdnih kopic. Z izdajo najnovejšega kataloga \Gs\ je določevanje njihovega članstva po večini dokaj trivialno in tako so že kmalu po izidu podatkov bili objavljeni prvi rezultati. Za naše potrebe raziskovanja razsutih kopic smo pridobili in uporabili rezultate pripadnosti kopic Berkeley 32, NGC 2516, NGC 2112, NGC 6253, Blanco 1, Ruprecht 147, NGC 2632, NGC 2682,  Melotte 22 in Collinder 261, ki jih je objavil \citet{2018A&A...618A..93C}. S podobno analizo smo sami določili še člane kopice Melotte 25, saj ta ni bila vsebovana v prej omenjenem delu.

\subsection{Raziskovanje okolice kopic}
V našem primeru nas je poleg samih znanih članov kopice zanimala še neposredna okolica kopice, saj zaradi dinamičnega dogajanja znotraj kopice le-ta počasi izgublja svoje člane, ki se pomešajo med okoliške zvezde. Izvržene članice nekaj časa še ohranijo kinematične podobnosti z matično kopico, kemična sestava pa na naj bi povezavo z rojstnim oblakom izkazovala cel njihov življenjski cikel. 

Za raziskovanje smo zato iz kataloga \Gs\ najprej pridobili vse dostopne informacije o zvezdah v širokem območju okrog središča kopice. Ker temnejše zvezde v tem katalogu nimajo popolne informacije o vektorju gibanja, smo jih dopolnili z radialnimi hitrostmi pridobljenimi v pregledu \Gh. S poznavanjem natančne hitrosti in galaktične lokacije zvezd kopice ter gravitacijskega potenciala Galaksije lahko z integracijo določimo pot posamezne zvezde po Galaksiji. Z uporabo programskega paketa \GP\ \cite{2015ApJS..216...29B} smo tako simulirali preteklo galaktično pot vseh zvezd s kompletnim naborom informacij za 120 milijonov let v preteklost, kar je primerljivo s starostjo najmlajših kopic v našem podatkovnem setu.

S poznavanjem poti trenutnih preostalih članov kopice smo ob vsakem vmesnem integracijskem koraku na njene najbolj zunanje člane napeli odsekoma ravne ploskve, ki so definirali najmanjši volumen za zajem celotne kopice. Za določitev v preteklosti možnih izvrženih članov smo poti vseh okoliških zvezd ob vsakem integracijskem koraku primerjali s takratnim volumnom kopice. Zaradi nedoločenosti \G\ parametrov smo vsako primerjavo izvedli 250-krat, kjer smo začetne parametre zvezd žrebali iz Gaussove porazdelitve, določene s srednjo vrednostjo in nedoločenostjo posameznega parametra. Iz vseh ponovitev smo določili verjetnost, da je bila okoliška zvezda nekoč izvržena iz kopice. Verjetnost določata njena pogostost prehoda volumna kopice ter čas tekom katerega se je najdlje zadrževala znotraj nje. V končni seznam potencialnih izvrženih so na koncu prišle le zvezde, ki so volumen kopice prečkale v vsaj $68$\% primerih ponovitve simulacije ter znotraj volumna najdlje ostale vsaj milijon let.

\subsection{Kemična sestava kopic in okolice}
Raziskano okolje okrog kopic smo tako razdelili na tri skupine zvezd: poznani člani kopice, potencialni v preteklosti izvrženi člani, ter naključne okoliške zvezde. Po teoriji bi te komponente lahko določili tudi samo z opazovanjem njihovih kemičnih podpisov. Za preverbo izvedljivosti smo najprej narisali grafe posameznih zastopanosti v odvisnosti od efektivne temperature zvezde - za prikaze glej Slike \ref{fig:ct_cluster1}, \ref{fig:ct_cluster2}, \ref{fig:ct_cluster3} in \ref{fig:ct_cluster4}. Prva stvar, ki nas je na njih zanimala, je seveda porazdelitev zastopanosti za znane člane kopic. Teoretični modeli pravijo, da bi zastopanosti morale biti identične za vse člane, ne glede na njihove fizikalne parametre. Tega ne vidimo v naših podatkih, saj zastopanost močno variira v odvisnosti od efektivne temperature zvezde. Efekt, ki je večji od same razpršenosti zastopanosti, je lahko vpogled v dejansko stanje ali pa odraža napake oziroma nepopolnosti tekom obdelave podatkov. Ker so trendi vse prisotni, gladki in različni med kopicami, se nam je drugi razlog zdel bolj realen. Za opis trendov smo na podatke zastopanosti v odvisnosti od efektivne temperature prilagodili poligon tretje stopnje.

Po prilagajanju smo se lotili diferencialnega postopka metode kemičnih podpisov, pri kateri smo vse zastopanosti primerjali z določenim trendom, ki naj bi opisoval stanje kemičnih elementov v opazovani kopici. Zanimalo nas je, koliko zvezd ozadja in potencialno izvrženih, ima kemični podpis podoben kopici. Podobnost smo določili tako, da smo za vsako zvezdo prešteli v kolikih elementih padejo njene vrednosti znotraj nedoločenosti okrog prilagojenega trenda. Zvezdo smo šteli kot kemično podobno, če se je s kopico ujemala vsaj v $68$\% elementov, ki smo jih opazovali (nekatere elemente smo pred analizo izpustili zaradi majhnega števila meritev). Rezultati za posamezne opazovane komponente so predstavljeni v Tabeli \ref{tab:cluster_stats_abundtag}.

\subsection{Rezultati in zaključki}
Raziskovane kopice in njihove zastopanosti dajejo vtis, da raziskovanje kemičnega prostora v okviru GALAH meritev ni enostavno in se je za poenostavitev metode kemičnih podpisov potrebno omejiti vsaj na ožji temperaturni razpon zvezd in omejen volumen Galaksije. Kljub izraženim trendom zastopanosti smo s predhodnim kinematičnim razločevanjem na komponente kopice poizkusili diferencialni pristop metode. Izkazalo se je, da bi brez predhodne kinematične informacije težko razločili člane nekaterih razsutih kopic, obenem pa so bili kinematično določeni potencialno izvrženi člani kemično podobnejši kopici kot naključne zvezde okrog nje.

% -----------------------------------------------------------------------
\section{Kemično posebne zvezde}
\label{sec:slo_c_peculiars}
Uspešnost omenjene metode kemičnih podpisov je močno odvisna tudi od tipa zvezde, saj nepričakovani posebni tipi spektrov lahko tekom obdelave spektra privedejo do napačno določenih fizikalnih parametrov zvezd in posledično tudi zastopanosti. V razsežnih neomejenih pregledih neba tako želimo čim bolj točno vedeti, ali naša opazovanja vsebujejo tudi takšne posebne zvezde. Tekom naše študije smo se zato osredotočili na nekaj zanimivih posebnih tipov. Prvi od njih so spektri, ki nakazujejo veliko vsebnost ogljika v atmosferi zvezde. Za njegovo identifikacijo smo uporabili široke molekularne SWAN pasove C$_{2}$, ki jih HERMES zajema na začetnem delu modrega spektralnega območja. 

\subsection{Nadzorovana klasifikacija}
Nepričakovane stvari v spektrih je najlažje najti tako, da jih primerjamo s setom referenčnih normalnih spektrov, ki opisujejo povprečni spekter poznanih zvezd. Tak nabor smo za vsako zvezdo posebej izdelali s povprečenjem spektrov, ki so z izbranim spektrom imeli zelo podobne fizikalne parametre ($\Delta$\Teff=$\pm75$~K, $\Delta$\Logg=$\pm0.125$~dex in $\Delta$\Feh=$\pm0.05$~dex). Spektralne razlike smo pridobili z deljenjem opazovanega in referenčnega spektra. Naša iskana spektralna lastnost se nahaja na že vnaprej znani lokaciji pri valovni dolžini 4737~\AA, kar smo izkoristili za določanje njene moči. Na predvideno območje smo prilagodili funkcijo
\begin{equation}
\centering
f(\lambda) = f_0 - \log{}\Gamma(\lambda, S, \lambda_0, A),
\label{equ:slo_loggamma}
\end{equation}
ki nam je sporočila obliko območja, njen integral pa stopnjo izraženosti ogljika v zvezdi. S pomočjo parametrov funkcije smo iz rezultatov najbolj obogatenih spektrov izločili možne napake obdelave podatkov.

\subsection{Nenadzorovana klasifikacija}
Poleg območja z najmočneje izraženim spektralnim pasom molekule C$_{2}$, ki smo ga uporabljali za nadzorovano določanje, se v njegovi okolici pojavlja še množica manj izrazitih identifikatorjev te molekule. Zanimalo nas je ali bi nenadzorovane metode strojnega učenja lahko samodejno prepoznale vse te značilnosti in naše iskane spektre razvrstile v skupno gručo. Za ta namen smo uporabili algoritem t-distributed Stochastic Neighbor Embedding (t-SNE, \cite{van2008visualizing}), ki deluje v dveh korakih. Najprej je med vsemi vhodnimi spektri (obrezani na območje 4720--4890~\AA) določil njihove medsebojne podobnosti z izračunom evklidskih razdalj. Na podlagi podobnosti algoritem poskuša najti optimalno preslikavo v dvo- ali tri-dimenzionalni prostor, ki je vizualno še sprejemljiv za človeka. Tekom transformacije dobimo posamezne skupine točk, ki predstavljajo podobne vhodne podatke. V našem primeru raziskovanja zvezd, bogatih z molekularnim ogljikom, je transformacija predstavljena na Sliki \ref{fig:tsne_plot}. Za lažje raziskovanje prikazane mape smo vanjo vrisali že prej zaznane kemično posebne spektre, kar nam je omogočilo lažje nadaljnjo pregledovanje tekom katerega smo odkrili še par zanimivih skupin zvezd. Te imajo poleg visoke vsebnosti ogljika še zelo nizko kovinskost, kar jih skupaj dela nadvse zanimive za nadaljnje študije.

\subsection{Rezultati in zaključki}
S kombinacijo nadzorovane in nenadzorovane metode klasifikacije smo odkrili 918 zvezd s kemično zanimivim spektrom. Z analizo njihovih fizikalnih parametrov, smo odkrili, da je večina teh zvezd orjakinj, manjši del pa pripada skupini pritlikavk. Čeprav manjšina v našem setu so zelo zanimive zaradi špekulacij glede njihovega točnega izvora obogatitve z ogljikom. Ena od možnosti predvideva, da je obogatitve opravila njena dvojna zvezda, ki se je že dolgo nazaj razvila, napihnila in s tem obogatila svojo sosedo. Znake takšnega sistema smo iskali na podlagi spreminjanja radialnih hitrosti, vendar z malim številom ponovljenih opazovanj nismo mogli potrditi obstoja kakšnega takega sistema. Za podrobnejšo analizo in potrditev bi potrebovali dodatna opazovanja, ki smo jih že pričeli izvajati na observatoriju Asiagu in v poglavju predstavili tudi enega izmed njih.

% -----------------------------------------------------------------------
\section{Emisijske zvezde}
\label{sec:slo_emisijske}
Podobno kot pri prejšnji določitvi posebnih tipov spektrov, smo tudi emisijske zvezde iskali z metodo direktne primerjave med normalnim oziroma referenčnim spektrom ter opazovanim spektrom. V spektru iskane značilke se nahajajo le v modrem in rdečem delu HERMES spektra, zato smo se osredotočiti le nanju.

\subsection{Simulacija spektrov z avtoenkoderjem}
Za izgradnjo referenčnih spektrov smo ponovno uporabili podatkovno usmerjeno metodo, ki lahko pokrije več dogajanja kot pa teoretični modeli. Odločili smo se za uporabo avtoenkoderja - posebne strukture nevronske mreže, ki s svojo sestavo vhodne podatke predela v dimenzijo mnogo manjšo od vhodne, nato pa jih po obratnem postopku razparkira in poskusi reproducirati vhodne podatke. Shema kodirnega dela strukture je predstavljena na Sliki \ref{fig:autoann}. Tako reproduciran izhodni signal je ponavadi zglajen signal brez spektralnih posebnosti in posnema povprečnost spektrov, ki so podobni vhodnemu. Vse to so lastnosti, ki si jih želimo od algoritma za izdelavo primerjalnih normalnih spektrov. 

Pred samo uporabo je bilo strukturo potrebno naučiti na naše podatke. Uporabili smo vse spektre, ki so imeli veljavne fizikalne parametre, ter izločili spektre že prej znanih posebnih tipov zvez \cite{2017ApJS..228...24T}, ki bi lahko onemogočali doseganje želenega učinka. Že majhen vnos posebnih tipov spektrov bi lahko povzročil, da se sistem nauči tudi njihovega izgleda in s tem reproducira tudi iskane značilke. Z zadostnim izločanjem nam je uspelo konstruirati sistem, ki tudi za posebne tipe spektrov vrne njim najboljši približek normalnega spektra, kar je prikazano na Slikah \ref{fig:refann} in \ref{fig:refann2}. Pridobljene vmesne skrčene značilke spektra se zelo dobro skladajo s fizikalnimi parametri zvezde (glejte Slike \ref{fig:latent_ccd3_1}, \ref{fig:latent_ccd1_1} in \ref{fig:latent_ccd1_2}), kar nakazuje, da se fizikalni model zvezdnega spektra in naš nenadzorovan podatkovno usmerjen model strinjata v najpomembnejših parametrih, ki pogojujejo izgled zvezdnega spektra.

\subsection{Določanje emisijskih komponent}
Po pridobitvi referenčnih spektrov za vsako zvezdo smo od njenega opazovanega spektra odšteli generiran referenčni spekter. V pridobljenem ostanku, ki poudarja njune razlike, smo se osredotočili na kromosferske spektralne emisije črt H$\alpha$ in H$\beta$, ter prepovedane prehode enkrat ioniziranih elementov [NII] in [SII], ki nam podajajo lastnosti razredčenega nebularnega plina v okolici zvezde. Vsak od elementov [NII] in [SII] v HERMES spektru izkazuje dve povezani emisijski črti. Za vodikovi črti smo določili naslednje parametre: ekvivalentno širino emisijskega dela, širino emisijske črte na 10\% njene največje moči ter z delnim integriranjem (premik v rdeči in modri del proč od mirovne valovne dolžine vodikove črte) še asimetričnost črte, ki nakazuje na njen izvor. Manj izrazite emisije [NII] in [SII] smo analizirali s hkratnim prilagajanjem dveh povezanih Gaussovih krivulj na izražene črte posameznega elementa. Tako smo za vsak element določili njegovo število izraženih črt, skupno ekvivalentno širino ter razliko radialne hitrosti glede na opazovano zvezdo.

\subsection{Rezultati in zaključki}
Z opisano analizo smo med vsemi spektralnimi podatki odkrili 10.364 spektrov z močneje izraženimi emisijskimi črtami v H$\alpha$/H$\beta$ območju in 4431 spektrov, ki dodatno vsebujejo še merljiv nebularni prispevek. Vse odkrite primere smo tudi narisali na zvezdno karto (glejte Sliki \ref{fig:spatialemission} in \ref{fig:spatialnebular}), kjer smo zaznali že vnaprej pričakovane korelacije pozicij zvezd z območji mladih zvezdnih kopic ter območji vidnega nebularnega medzvezdnega plina. Tekom analize in nadzora kvalitete rezultatov smo kot obstranski produkt izdelali še seznam dvojnih zvezd z izraženimi dvojnimi spektralnimi črtami v spektru ter označili potencialne nepravilnosti pri odštevanju ozadja tekom redukcije podatkov. Med vsemi opazovanji smo našli tudi 621 zvezd z dvema ali več opazovanji, pri katerih smo za vsaj enega potrdili prisotnost emisijskih črt. Večina ponovljenih opazovanj ne kaže spreminjanja oblike ali lokacije profila, preostali spreminjajoči objekti pa predstavljajo zanimiv podatkovni set za nadaljnje raziskave.

% -----------------------------------------------------------------------
\section{Soncu podobne večkratne zvezde}
\label{sec:slo_soncevi_veckratniki}
Izmed vseh opazovanih zvezd na nebu nam je Sonce najbolj poznano, saj njegova bližina in svetlost omogočata podrobne spektroskope, fotometrične, strukture in druge analize. Zaradi tega je Sonce uporabljeno kot referenčna zvezda za veliko fizikalnih meritev. Poznavanje večjega števila Soncu skoraj identičnih zvezd v velikih pregledih neba nam omogoča njihovo notranje umerjanje \cite{2010A&A...522A..98M, 2012MNRAS.426..484D} in medsebojno primerjavo. Zaradi tega smo se odločili, da tudi v pregledu GALAH poiščemo čim več takih zvezd, katerih spekter je čim bolj identičen sončevemu.

\subsection{Izbira Soncu najbolj podobnih zvezd}
Izbiro zvezd smo pričeli z izdelavo sončevega spektra, kot ga posnamemo s spektrografom HERMES. Tekom rednih opazovanj, za namen umerjanj, smo posneli tudi spektre neba ob sončevem vzhodu in zahodu, ki odražajo točen spekter Sonca. Z njihovim povprečenjem smo izdelali skoraj brezšumni referenčni spekter Sonca. Najbolj identične opazovano spektre smo izbrali tako, da smo v vsakem valovnem območju HERMES spektrografa izračunali razdaljo Canberra \cite{Lance1967MixedDataCP} med referenčnim in opazovanim spektrom. Tako smo dobili štiri neodvisne cenilke podobnosti za vsak spekter. Izračunan podobnost je močno odvisna od razmerja med signalom in šumom opazovanega spektra. V primeru enoznačno izbranega pragu za izbiro bi tako pridobili le spektre z najmanj šuma, ostale pa zanemarili, čeprav bi morda dejansko bili podobnejši Soncu. V izogib problemu smo v izbiro vključili tudi moč šuma - primer postopka na Sliki \ref{fig:envelope_fit_twins}. V končni seznam smo uvrstili 329 spektrov, ki so bili v vseh štirih valovnih pod-območjih razvrščeni med 7\% najpodobnejših spektrov.

\subsection{Določanje večkratnosti}
Za izbran set zvezd bi pričakovali, da je njihov absoluten izsev zelo podoben oziroma identičen sončevemu. Z združevanjem \G\ podatkov navidezne magnitude in oddaljenosti \cite{2018AJ....156...58B} smo na Sliki \ref{fig:par_gmean} ugotovili, da nekatere zvezde izsevajo tudi dva- in več-kratnik sončeve svetlobe. Ker v samem spektru ni opaznih podvojenih spektralni črt, ki bi nakazovale na večkratnost objekta, smo predvidevali, da morajo biti ti sistemi sestavljeni iz skoraj identičnih zvezd na dokaj veliki medsebojni oddaljenosti z orbitalno hitrostjo, ki ne povzroči delitve spektralnih črt.

Simulacijo možnih kombinacij, ki bi nam podale opazovane vrednosti, smo razdelili na spektralni in orbitalni del. V spektralnem delu smo iz množice opazovalnih podatkov sestavili podatkovno usmerjena modela, ki sta neodvisno opisovala fotometrični in spektralni podpis enojne zvezde. S postopnim sestavljanjem naključnih kombinacij (metoda Monte Carlo markovske verige oziroma MCMC, \cite{2013PASP..125..306F}) dveh in treh različnih enojnih zvezd smo poskusili poustvariti opazovane fotometrične in spektralne podatke. Tako smo potrdili, da bi opazovane podatke s končnimi rezultati, obarvanimi na Sliki \ref{fig:gabs_binmulti}, lahko sestavili iz več zvezdnih komponent.

V orbitalnem delu analize smo želeli preveriti ali simulirani objekti tudi res lahko obstajajo znotraj opazovalnih omejite. Pri tem smo predpostavili, da je naj množičen trojni hierarhičen sistem sestavljen iz dveh blizu krožečih objektov in enega oddaljenega. S simulacijo podobnosti spektrov, pri katerem je ena od komponent zamaknjena za določeno radialno hitrost, smo določili največjo možno hitrost, pri kateri se izmerjena podobnost ne spremeni preveč da sestavljen spekter ne bi bil več podoben Soncu. To hitrost smo uporabili za določitev najmanjše velikosti velike polosi orbite bližnjih objektov. Tak sistem je možen, saj je najmanjša velikost dosti manjša od mejne največje oddaljenosti zunanje zvezde. Ta je določena z mejno ločljivostjo satelita \G, pri kateri še lahko razloči dva zvezdna izvora svetlobe.

Možnost orbitalnih dinamik smo preveri še z vključitvijo radialnih hitrosti zbranih iz pregledov \G, RAVE \cite{2017AJ....153...75K}, GALAH in posameznih opazovanj Asiago zvezd. Z združevanjem katalogov smo tako pridobili dodatno časovno dinamiko objekta, vendar za posamezno zvezdo nikoli nismo pridobili več kot treh časovnih točk. Tako majhno število onemogoča kompleksne analize, vendar nam vseeno podaja spodnjo mejo spremenljivosti hitrosti objekta. Med zvezdami nismo našli definitivnega spremenljivega izvora, saj so bile največje spremembe v rangu 5~\kms in manj, kar je primerljivo z nedoločenostjo teh meritev.

\subsection{Rezultati in zaključki}
Uporabljena analiza identificiranih sončevih dvojnikov temelji na podatkih oddaljenosti, ki se bodo s prihodnjimi podatkovnimi izidi še spreminjali in potencialno zelo spremenili rezultate. Kljub temu nam izsledki nakazujejo, da je pri uporabi identičnih spektrov kljub temu potrebno biti zelo pozoren, saj so kljub njihovem izgledu lahko sestavljeni iz več skoraj identičnih zvezd na za opazovalca počasnih medsebojnih orbitah. V našem primeru so bili simulirana pogoji takšnih orbitalnih konfiguracij znotraj opazovalnih omejitev in posledično identificirani sistemi tudi fizikalno mogoči. Ker nas je zanimalo ali podobno velja tudi za bolj vroče in hladnejše zvezde na glavni veji, smo analizo večkratnost identičnih spektrov razširili še na njih. Rezultati na Sliki \ref{fig:triple_hr} nakazujejo že poznan trend \cite{2013ARA&A..51..269D} povečevanja razširjenosti večkratnih sistemov med bolj vročimi zvezdami.

% -----------------------------------------------------------------------
\section{Zaključki disertacije in prihodnje študije}
\label{sec:slo_zakljucek}
Z napredkom opazovalnih metod in observatorijev se počasi spreminja tudi opazovalno delo astronomije. Očiten premik se vrši v smeri proč od dolgotrajnih analiz posameznih objektov proti masovnim pregledom neba in temu primernim obdelavam. Tega človek seveda ne more sam, zato razvija množico računalniških algoritmov, ki mu pomagajo pri tej nalogi. Med njimi trenutno najhitreje rastoči po uporabi in včasih tudi napačno uporabljeni, so algoritmi strojnega učenja, ki upravljajo naloge klasifikacije, grozdenja in regresije.

V disertaciji smo uporabili nekaj od teh orodij za raziskovanje različnih astronomskih podatkovnih setov, ki so bili primarno pridobljeni kot del pregledov neba \Gh\ in \G. Vsi pridobljeni rezultati so zainteresiranim brezplačno dostopni na spletnem mestu VizieR \footnote{\url{http://vizier.u-strasbg.fr/}} in na straneh založnikov objavljenih znanstvenih člankov. S pridobitvijo dodatnih opazovanj, predvsem časovne komponente, objavljeni rezultati ponujajo še kopico možnosti za nadaljnje študije, ki bi se poglobile v fizikalna ozadja razpoznanih posebnosti. Takšne poglobljene raziskave se ponavadi izvajajo individualno za vsak objekt posebej in potrebujejo čim širši nabor opazovalnih podatkov.

Novih podatkovnih setov v bližnji prihodnosti ne bo zmanjkalo, saj se ravno v zadnjih letih vršijo končne faze priprave in izdelave več masovnih fotometričnih in spektroskopskih pregledov neba. Do njihovega začetka pa nestrpno pričakujemo še tretji in za tem posledično končni izid podatkov \G, ki še lahko spremenijo in dopolnijo naše vedenje o sestavi in razvoju Galaksije.