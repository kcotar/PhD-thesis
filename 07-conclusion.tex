The fast increase in observational data seen in the last decade, with the introduction of new and improved astronomical observational facilities, requires new approaches to the exploration of acquired data. The old approach of exact and painstaking analysis of individual objects have to be changed in order to grasp the full potential of large observational sets. Consequently, large amounts of data and complex observational scenarios also lead to more complicated data reduction and analysis pipelines. As it is impossible to look through all acquired data and consider all observational and physical variables, many computer algorithms have been developed over the years to help with those tasks. The most trending and occasionally misused are numerous machine learning procedures of classification, clustering, and regression.

In this thesis, we used some of the available machine learning tools to explore large astronomical datasets, mainly collected as part of the \Gh\ and \G\ surveys. The data of those surveys were, when needed, supplemented with results of other spectroscopic and photometric surveys. This merger gave us additional information by observing a broader wavelength range and an increase in temporal coverage of variabilities by combing similar data sets.

The body of this thesis focuses on exploring different types of stars. Usually, the stars are generally separated into two broad groups: normal and peculiar. The first term describes the majority of the stars that are happily living through the longest periods of their lifetime. Being non-problematic, they are easily modelled and preferred for chemical analyses such as our exploration of open stellar clusters and their surrounding. During the analysis, we explored the possibility of chemical differentiation between chemical signatures of cluster members, possible past members and surrounding stars. Separation into given components for 11 open clusters was based on kinematic vectors defined by \G\ observations. The \Gh\ survey supplied the chemical signature for a subset of stars. The analysis of open clusters in our dataset showed that they are not as chemically homogeneous as theories say. Not knowing the origin of this discrepancy, we shifted to the differential chemical analysis that takes out the identified trends and compares only chemical signatures of stars with the same stellar parameters, most commonly the effective temperature. The results show that the uncertainties of the determined abundances and their scatter limit chemical separability between an open cluster and field stars as the majority have very comparable signatures. We showed that inclusion of kinematic information helps with the delineation as it was more likely that potentially ejected stars were chemically tagged to the cluster than remaining field stars.

The second large group of stars that are not wanted in the above analyses are peculiar stars. The term is inclusive and depends on the scientific question in hand and observed wavelength region. In our case, the definition includes spectroscopically detectable classes such as interacting stars, multiple stellar systems, stars in temporally short-lived evolutionary stages, and spectra with rare chemical compositions. Our selection of all peculiar classes depended on the comparison between normal-looking spectra and observed spectra. The modelling of normal spectra was done in multiple ways: by averaging spectra with similar parameters, running observed spectra trough a neural network autoencoder and modelling using spectrum generative approach called \TC. The direct comparison gave us a difference between spectral modelling and reality. When we were looking at the specific wavelength regions, we uncovered spectra with pronounced molecular absorption bands of C$_2$ - stars that are carbon-rich and spectra with expressed hydrogen, [NII], and [SII] emission lines. As the emission lines are results of different ongoing processes on a star and around it, we tried to separate them into chromospheric and nebular contributions using the derived line parameters such as position, strength and shape. Combination of normal single star spectra can also be used for the spectral disentanglement of multiple stellar systems. Our methodology was used to describe over-luminous stars whose spectra did not show signs of multiple stars in a system. Therefore we assumed that comprising stars in the system must be orbiting on wide orbits. By combining photometric and spectroscopic single star models, we were able to reproduce the observable quantities and confirm the existence of triple systems with near-identical stars. Orbits of such proposed systems were also modelled to confirm that their configuration is consistent with observational limitations.

All tabulated results presented in this thesis are also published as freely available catalogues on the astronomical online database collection VizieR and directly from the publishers' website. The compiled catalogues could serve as a starting point for diverse additional lines of research which focus on the exact physics behind identified peculiar stars and their spectra. Some of the possibilities for future studies were already given in the text and will be considered as potential observing possibilities at the available observing facilities, such as Asiago observatory where we are conducting spectroscopic observations almost every month. Most often, spectroscopic data must be complemented with photometric sets to explore or confirm additional possible physical scenarios behind interesting objects.

The hype of the big data era in astronomy is still to increase as many new telescopes and instruments are currently under development and construction. Their observations are planned to start in the following years. Until then, the \G\ satellite is continuously scanning our sky in order to bring us the most precise distances and movements for billions of stars that we are all waiting for. The next grander data release is still at least a year away, but everyone is already questioning how much more can it do for science in the field of galactic archaeology. In our case, the updated distances could completely change the membership, shape and structure of open clusters and reclassify exciting possible triple stars to dull single stars because of their parallactic uncertainty. Until then, we have to double-check our applied quality filters and trust in the currently obtainable data and parameters. The journey into big astronomical data does not seem to be stopping anywhere in the coming years.