The fast increase in observational data seen in the last decade with the introduction of new and improved astronomical observational facilities requires new approaches to the exploration of acquired data. The old mentally of exact and painstaking analysis of individual objects have to be changed in order to grasp the full potential of large observational sets. On the other hand, large amounts of data and complex observational scenarios also lead to more complicated data reduction and analysis pipelines. As it is impossible to look through all acquired data and consider all variables, many computer algorithms have over the years been developed to help with those tasks. Of them, the most trending and occasionally misused are machine learning procedures of classification, clustering, and regression.

In this thesis, we have used some of the machine learning tools to explore large astronomical datasets, mainly collected as part of the \Gh\ and \G\ sky surveys. \ldots