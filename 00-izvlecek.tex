%Naraščajoče število popolnoma avtomatiziranih in visoko multipleksiranih teleskopov daje vedno večje število opazovanj, katerih zmanjšanje in analiza sta čim bolj avtomatizirana. Ker omejeno število raziskovalcev ne more ročno obdelati in preveriti vseh zbranih informacij, se bo ta stopnja avtomatizacije s časom stalno povečevala. To ne pomeni, da so uporabljeni procesi brezhibni in bo znanost v bližnji prihodnosti predana strojem. Ravno nasprotno, zapletenost strojev in raziskovanih znanstvenih tem še vedno zahteva, da človek prilagodi procese, ugotovi prej nepriznane težave in razmišlja o naslednjem znanstvenem problemu.
%
%Orodja za strojno učenje niso čarobne rešitve, saj le skušajo rešiti težavo, ki jo je uvedel upravljavec. Če je vprašanje slabo opredeljeno ali so podatki nepravilno pripravljeni in filtrirani, bodo te odločitve obremenjene tudi z rezultati. To predstavlja pomembno potrebo po tem, da ne poznamo samo razpoložljivih podatkov, ampak tudi, kako so bili pripravljeni in morda kakovostno označeni med proizvodnjo.
%
%Ta disertacija predstavlja rezultate našega raziskovanja obsežne podatkovne baze GALactic Archaeology with HERMES (\Gh) z različnimi orodji strojnega učenja in poudarja potrebo po nadzoru in razumevanju vhodnih podatkov. Predstavljeni rezultati obravnavajo raziskovanje odprtih zvezdnih grozdov in njihov kemijski podpis, pri čemer sta glavna problema homogenost in pravilnost določenih kemičnih sestavov. Z razumevanjem omejitev kemično označevanja odprtih zvezdnih grozdov s homogeno sestavo so bile ugotovitve uporabljene pri kemičnem raziskovanju njihove okolice.
%
%Drugi večji del teze se ukvarja z iskanjem manjših podskupov svojevrstnih spektrov v podatkovni bazi \Gh. Z različnimi pristopi primerjave med opazovanimi in modeliranimi normalnimi spektri smo iskali spektre z izrazitimi molekularnimi pasovi molekule C$_2 $, emisijskimi črtami in spektroskopsko nerazpoznavne večkratne sisteme.

\rb{TODO}\\[10mm]
{\bf Ključne besede:} metode: analiza podatkov -- razsute kopice in asociacije: splošno -- Galaksija: zvezdne populacije -- zvezde: zastopanosti -- zvezde: hitrosti in premikanje -- dvojnice: splošno -- zvezde: posebni tipi -- zvezde: karbonske -- zvezde: aktivnost -- zvezde: emisijske črte -- zvezde: podobne Soncu -- črte: profili -- katalogi\\[3mm]
{\bf PACS:} 95.10.Eg, 95.75.De, 95.75.Fg, 95.75.Mn, 95.80.+p, 97.10.-q, 97.10.Ri, 97.10.Tk, 97.10.Vm, 97.21.+a, 97.30.Eh, 97.30.Fi, 97.80.-d, 97.80.Fk, 98.58.H