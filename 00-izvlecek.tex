Naraščajoče število popolnoma avtomatiziranih in visoko multipleksiranih teleskopov nam posreduje vedno večje število opazovanj, katerih začetna obdelava in analiza sta vedno bolj in bolj avtomatizirani. Ker omejeno število raziskovalcev ne more ročno obdelati in preveriti vseh zbranih informacij, se bo stopnja avtomatizacije s časom le še povečevala. To ne pomeni, da so uporabljeni procesi brezhibni in bo znanost v bližnji prihodnosti predana računalnikom. Ravno nasprotno, zapletenost teleskop in opazovanj ter  znanstvenih vprašanj še vedno zahteva, da operator prilagodi procese, ugotovi prej neopažene težave in razmišlja o naslednjem znanstvenem problemu.

Orodja za strojno učenje niso čarobna orodja, saj le skušajo rešiti težavo, ki si jo je zamislil njihov uporabnik. Če je vprašanje slabo opredeljeno ali so podatki nepravilno pred-pripravljeni in filtrirani, bodo tudi rezultati obremenjeni z vhodnimi odločitvami. To predstavlja pomembno težnjo po poznavanju ne samo razpoložljivih podatkov, ampak tudi kako so bili pripravljeni in označeni z raznimi statusi njihove kvalitete.

Disertacija predstavlja rezultate našega raziskovanja obsežne podatkovne baze GALactic Archaeology with HERMES (\Gh) z različnimi orodji strojnega učenja in poudarja potrebo po nadzoru in razumevanju vhodnih podatkov. Predstavljeni rezultati obravnavajo raziskovanje 16 razsutih zvezdnih kopic in njihov kemični podpis. Pri tem sta glavna problema homogenost in pravilnost izračunanih kemičnih zastopanosti. Pri kemičnem raziskovanju okolice razsutih kopic smo uporabili razumevanje omejitev uporabe metode kemičnih podpisov razsutih zvezdnih kopic, ki naj bi imele homogeno sestavo.

Drugi večji sklop disertacije se ukvarja z iskanjem manjših podskupin posebnih tipov spektrov v podatkovni bazi \Gh. Z različnimi pristopi primerjave med opazovanimi in modeliranimi normalnimi spektri smo našli $918$ spektrov z izrazitimi molekularnimi pasovi molekule C$_2$, $10.364$ spektrov z emisijskimi črtami ter analizirali $329$ kandidatov za spektroskopsko nerazpoznavne večkratne sisteme.\\[10mm]
{\bf Ključne besede:} metode: analiza podatkov -- razsute kopice in asociacije: splošno -- Galaksija: zvezdne populacije -- zvezde: zastopanosti -- zvezde: hitrosti in premikanje -- dvojnice: splošno -- zvezde: posebni tipi -- zvezde: karbonske -- zvezde: aktivnost -- zvezde: emisijske črte -- zvezde: podobne Soncu -- črte: profili -- katalogi\\[3mm]
{\bf PACS:} 95.10.Eg, 95.75.De, 95.75.Fg, 95.75.Mn, 95.80.+p, 97.10.-q, 97.10.Ri, 97.10.Tk, 97.10.Vm, 97.21.+a, 97.30.Eh, 97.30.Fi, 97.80.-d, 97.80.Fk, 98.58.H